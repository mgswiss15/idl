\documentclass[11pt,a4paper]{article}
\usepackage[margin=2.5cm]{geometry}
\usepackage{amsmath,amssymb}
\usepackage{enumitem}
\usepackage{hyperref}
\usepackage{listings}
\usepackage{xcolor}
\usepackage{tcolorbox}
\usepackage{fontawesome5}

% Code styling
\lstset{
    basicstyle=\ttfamily\small,
    keywordstyle=\color{blue},
    commentstyle=\color{gray},
    stringstyle=\color{red},
    showstringspaces=false,
    breaklines=true,
    frame=single,
    backgroundcolor=\color{gray!10}
}

% Custom boxes
\newtcolorbox{problembox}{
    colback=red!5!white,
    colframe=red!75!black,
    title=\faBug\ Problem
}

\newtcolorbox{solutionbox}{
    colback=green!5!white,
    colframe=green!75!black,
    title=\faCheck\ Solution
}

\newtcolorbox{warningbox}{
    colback=orange!5!white,
    colframe=orange!75!black,
    title=\faExclamationTriangle\ Warning
}

\title{Introduction to Deep Learning\\
Troubleshooting Guide}
\author{MSc Computer Science}
\date{Week 1}

\begin{document}

\maketitle

\tableofcontents
\newpage

\section{Introduction}

This guide covers common problems encountered during environment setup and how to fix them. Problems are organized by operating system and category.

\section{General Issues}

\subsection{``command not found'' or ``is not recognized''}

\begin{problembox}
When trying to run \texttt{python} or \texttt{pip}, you get:
\begin{itemize}
    \item Windows: ``'python' is not recognized as an internal or external command''
    \item Linux/macOS: ``python: command not found''
\end{itemize}
\end{problembox}

\begin{solutionbox}
\textbf{Windows:}
\begin{enumerate}
    \item Python is not in your PATH
    \item Reinstall Python and check ``Add Python to PATH''
    \item Or manually add Python to PATH:
    \begin{itemize}
        \item Search for ``Environment Variables'' in Start Menu
        \item Click ``Environment Variables''
        \item Under ``System variables'', find ``Path'', click ``Edit''
        \item Click ``New'' and add: \texttt{C:\textbackslash Users\textbackslash YourUsername\textbackslash AppData\textbackslash Local\textbackslash Programs\textbackslash Python\textbackslash Python311}
        \item Also add: \texttt{C:\textbackslash Users\textbackslash YourUsername\textbackslash AppData\textbackslash Local\textbackslash Programs\textbackslash Python\textbackslash Python311\textbackslash Scripts}
        \item Click OK, close all terminals, open a new one
    \end{itemize}
\end{enumerate}

\textbf{Linux/macOS:}
\begin{enumerate}
    \item Try \texttt{python3} instead of \texttt{python}
    \item If still not found, reinstall Python (see setup guide)
    \item Check installation:
    \begin{lstlisting}[language=bash]
which python3
    \end{lstlisting}
\end{enumerate}
\end{solutionbox}

\subsection{Virtual Environment Not Activating}

\begin{problembox}
After running the activation command, you don't see \texttt{(deep\_learning\_env)} in your prompt.
\end{problembox}

\begin{solutionbox}
\textbf{Windows (PowerShell):}
\begin{enumerate}
    \item PowerShell may block scripts by default
    \item Run PowerShell as Administrator
    \item Execute:
    \begin{lstlisting}[language=bash]
Set-ExecutionPolicy -ExecutionPolicy RemoteSigned -Scope CurrentUser
    \end{lstlisting}
    \item Try activating again:
    \begin{lstlisting}[language=bash]
deep_learning_env\Scripts\Activate.ps1
    \end{lstlisting}
\end{enumerate}

\textbf{Windows (Command Prompt):}
\begin{enumerate}
    \item Make sure you're using \texttt{.bat} not \texttt{.ps1}:
    \begin{lstlisting}[language=bash]
deep_learning_env\Scripts\activate.bat
    \end{lstlisting}
\end{enumerate}

\textbf{Linux/macOS:}
\begin{enumerate}
    \item Make sure you use \texttt{source}:
    \begin{lstlisting}[language=bash]
source deep_learning_env/bin/activate
    \end{lstlisting}
    \item Check if venv was created successfully:
    \begin{lstlisting}[language=bash]
ls deep_learning_env/bin/
    \end{lstlisting}
    \item You should see \texttt{activate} file
\end{enumerate}
\end{solutionbox}

\subsection{Permission Denied Errors}

\begin{problembox}
Getting ``Permission denied'' when trying to create virtual environment or install packages.
\end{problembox}

\begin{solutionbox}
\textbf{Linux/macOS:}
\begin{enumerate}
    \item DON'T use \texttt{sudo} with pip or venv
    \item Make sure you own the directory:
    \begin{lstlisting}[language=bash]
cd ~/Documents
mkdir DeepLearning
cd DeepLearning
    \end{lstlisting}
    \item If you accidentally used sudo, fix permissions:
    \begin{lstlisting}[language=bash]
sudo chown -R $USER:$USER ~/Documents/DeepLearning
    \end{lstlisting}
\end{enumerate}

\textbf{Windows:}
\begin{enumerate}
    \item Don't create the folder in \texttt{C:\textbackslash Program Files}
    \item Use your user directory: \texttt{C:\textbackslash Users\textbackslash YourUsername\textbackslash Documents}
\end{enumerate}
\end{solutionbox}

\section{Package Installation Issues}

\subsection{``No module named 'pip''}

\begin{problembox}
Error when trying to use pip: \texttt{No module named 'pip'}
\end{problembox}

\begin{solutionbox}
\textbf{Windows:}
\begin{lstlisting}[language=bash]
python -m ensurepip --upgrade
\end{lstlisting}

\textbf{Linux:}
\begin{lstlisting}[language=bash]
sudo apt install python3-pip
\end{lstlisting}

\textbf{macOS:}
\begin{lstlisting}[language=bash]
python3 -m ensurepip --upgrade
\end{lstlisting}
\end{solutionbox}

\subsection{PyTorch Installation Fails or Takes Forever}

\begin{problembox}
\texttt{pip install torch} is extremely slow or fails with timeout errors.
\end{problembox}

\begin{solutionbox}
\begin{enumerate}
    \item Use PyTorch's own package index (faster):
    \begin{lstlisting}[language=bash]
pip install torch torchvision torchaudio --index-url https://download.pytorch.org/whl/cpu
    \end{lstlisting}
    
    \item If still slow, try increasing timeout:
    \begin{lstlisting}[language=bash]
pip install --default-timeout=1000 torch torchvision torchaudio
    \end{lstlisting}
    
    \item Or download wheel file manually:
    \begin{itemize}
        \item Go to \url{https://download.pytorch.org/whl/torch_stable.html}
        \item Download appropriate \texttt{.whl} file for your system
        \item Install with:
        \begin{lstlisting}[language=bash]
pip install path/to/downloaded/file.whl
        \end{lstlisting}
    \end{itemize}
\end{enumerate}
\end{solutionbox}

\subsection{SSL Certificate Errors}

\begin{problembox}
\texttt{SSL: CERTIFICATE\_VERIFY\_FAILED} when installing packages.
\end{problembox}

\begin{solutionbox}
\textbf{Quick fix (not recommended for production):}
\begin{lstlisting}[language=bash]
pip install --trusted-host pypi.org --trusted-host pypi.python.org --trusted-host files.pythonhosted.org -r requirements.txt
\end{lstlisting}

\textbf{Proper fix:}
\begin{enumerate}
    \item Update certificates:
    \begin{itemize}
        \item Windows: Run \texttt{Install Certificates.command} in Python installation directory
        \item macOS: 
        \begin{lstlisting}[language=bash]
/Applications/Python\ 3.11/Install\ Certificates.command
        \end{lstlisting}
        \item Linux:
        \begin{lstlisting}[language=bash]
sudo apt install ca-certificates
        \end{lstlisting}
    \end{itemize}
\end{enumerate}
\end{solutionbox}

\subsection{Version Conflicts}

\begin{problembox}
``Could not find a version that satisfies the requirement'' or dependency conflicts.
\end{problembox}

\begin{solutionbox}
\begin{enumerate}
    \item Make sure virtual environment is activated
    \item Update pip:
    \begin{lstlisting}[language=bash]
pip install --upgrade pip
    \end{lstlisting}
    
    \item Try installing packages one by one instead of from requirements.txt:
    \begin{lstlisting}[language=bash]
pip install torch
pip install jupyter
pip install matplotlib
pip install numpy
    \end{lstlisting}
    
    \item If still failing, create a new virtual environment:
    \begin{lstlisting}[language=bash]
deactivate
rm -rf deep_learning_env  # or delete folder on Windows
python3 -m venv deep_learning_env
source deep_learning_env/bin/activate
pip install --upgrade pip
pip install -r requirements.txt
    \end{lstlisting}
\end{enumerate}
\end{solutionbox}

\section{Jupyter Notebook Issues}

\subsection{Jupyter Notebook Won't Start}

\begin{problembox}
\texttt{jupyter notebook} command fails or browser doesn't open.
\end{problembox}

\begin{solutionbox}
\begin{enumerate}
    \item Make sure Jupyter is installed:
    \begin{lstlisting}[language=bash]
pip install jupyter notebook
    \end{lstlisting}
    
    \item Try specifying a port:
    \begin{lstlisting}[language=bash]
jupyter notebook --port=8889
    \end{lstlisting}
    
    \item Check if port 8888 is already in use:
    \begin{itemize}
        \item Windows:
        \begin{lstlisting}[language=bash]
netstat -ano | findstr :8888
        \end{lstlisting}
        \item Linux/macOS:
        \begin{lstlisting}[language=bash]
lsof -i :8888
        \end{lstlisting}
    \end{itemize}
    
    \item If browser doesn't open, manually copy the URL from terminal (starts with \texttt{http://localhost:8888/...})
\end{enumerate}
\end{solutionbox}

\subsection{Kernel Won't Connect or Keeps Dying}

\begin{problembox}
Jupyter kernel fails to start or crashes when running cells.
\end{problembox}

\begin{solutionbox}
\begin{enumerate}
    \item Install/reinstall ipykernel:
    \begin{lstlisting}[language=bash]
pip install --upgrade ipykernel
python -m ipykernel install --user --name=deep_learning_env
    \end{lstlisting}
    
    \item Select correct kernel in Jupyter:
    \begin{itemize}
        \item In notebook: Kernel → Change kernel → deep\_learning\_env
    \end{itemize}
    
    \item Check for memory issues (especially on older computers):
    \begin{itemize}
        \item Close other applications
        \item Try smaller batch sizes in code
    \end{itemize}
    
    \item Restart Jupyter completely:
    \begin{lstlisting}[language=bash]
# In terminal where Jupyter is running: Ctrl+C, Ctrl+C
jupyter notebook
    \end{lstlisting}
\end{enumerate}
\end{solutionbox}

\subsection{Import Errors in Jupyter}

\begin{problembox}
\texttt{ModuleNotFoundError: No module named 'torch'} even though you installed it.
\end{problembox}

\begin{solutionbox}
Jupyter is using wrong Python environment!

\begin{enumerate}
    \item Check which Python Jupyter is using:
    \begin{lstlisting}[language=Python]
import sys
print(sys.executable)
    \end{lstlisting}
    Should point to your \texttt{deep\_learning\_env}
    
    \item Make sure virtual environment was activated before starting Jupyter
    
    \item Install Jupyter inside the virtual environment:
    \begin{lstlisting}[language=bash]
# Activate environment first!
source deep_learning_env/bin/activate  # Linux/macOS
# or: deep_learning_env\Scripts\activate  # Windows

pip install jupyter
jupyter notebook
    \end{lstlisting}
    
    \item Register the environment as a kernel:
    \begin{lstlisting}[language=bash]
python -m ipykernel install --user --name=deep_learning_env --display-name="Python (Deep Learning)"
    \end{lstlisting}
    Then in Jupyter: Kernel → Change kernel → Python (Deep Learning)
\end{enumerate}
\end{solutionbox}

\section{Platform-Specific Issues}

\subsection{Windows Specific}

\subsubsection{Long Path Issues}

\begin{problembox}
Errors about path names being too long.
\end{problembox}

\begin{solutionbox}
\begin{enumerate}
    \item Enable long paths in Windows:
    \begin{itemize}
        \item Open Registry Editor (Win+R, type \texttt{regedit})
        \item Navigate to: \texttt{HKEY\_LOCAL\_MACHINE\textbackslash SYSTEM\textbackslash CurrentControlSet\textbackslash Control\textbackslash FileSystem}
        \item Set \texttt{LongPathsEnabled} to 1
    \end{itemize}
    
    \item Or create project in shorter path:
    \begin{lstlisting}[language=bash]
cd C:\DL
    \end{lstlisting}
\end{enumerate}
\end{solutionbox}

\subsubsection{Antivirus Blocking Installation}

\begin{problembox}
Installation hangs or fails with cryptic errors.
\end{problembox}

\begin{solutionbox}
\begin{enumerate}
    \item Temporarily disable antivirus during installation
    \item Add Python and pip to antivirus exceptions
    \item Use Windows Defender instead of third-party antivirus if possible
\end{enumerate}
\end{solutionbox}

\subsection{macOS Specific}

\subsubsection{``xcrun: error'' on macOS}

\begin{problembox}
\texttt{xcrun: error: invalid active developer path}
\end{problembox}

\begin{solutionbox}
Install Xcode Command Line Tools:
\begin{lstlisting}[language=bash]
xcode-select --install
\end{lstlisting}
\end{solutionbox}

\subsubsection{M1/M2/M3 Mac Issues}

\begin{problembox}
Architecture errors or package incompatibilities on Apple Silicon.
\end{problembox}

\begin{solutionbox}
\begin{enumerate}
    \item Make sure you're using ARM version of Python, not x86:
    \begin{lstlisting}[language=bash]
python3 -c "import platform; print(platform.machine())"
    \end{lstlisting}
    Should print \texttt{arm64}, not \texttt{x86\_64}
    
    \item If you installed Python via Homebrew:
    \begin{lstlisting}[language=bash]
arch -arm64 brew install python@3.11
    \end{lstlisting}
    
    \item Some packages may need Rosetta 2:
    \begin{lstlisting}[language=bash]
softwareupdate --install-rosetta
    \end{lstlisting}
\end{enumerate}
\end{solutionbox}

\subsection{Linux Specific}

\subsubsection{``python3-venv'' not found}

\begin{problembox}
\texttt{The virtual environment was not created successfully because ensurepip is not available}
\end{problembox}

\begin{solutionbox}
Install venv module:
\begin{lstlisting}[language=bash]
# Ubuntu/Debian
sudo apt install python3.11-venv

# Fedora
sudo dnf install python3-virtualenv

# Arch
sudo pacman -S python-virtualenv
\end{lstlisting}
\end{solutionbox}

\subsubsection{``externally-managed-environment''}

\begin{problembox}
Error: ``externally-managed-environment'' when trying to install packages (common on Ubuntu 23.04+).
\end{problembox}

\begin{solutionbox}
This is why we use virtual environments! Make sure you:
\begin{enumerate}
    \item Created a virtual environment
    \item Activated it before installing packages
    \item See \texttt{(deep\_learning\_env)} in your prompt
\end{enumerate}

If you really need to install globally (not recommended):
\begin{lstlisting}[language=bash]
pip install --break-system-packages package_name
\end{lstlisting}
But use virtual environments instead!
\end{solutionbox}

\section{Performance Issues}

\subsection{Installation is Extremely Slow}

\begin{problembox}
Package installation takes hours or appears stuck.
\end{problembox}

\begin{solutionbox}
\begin{enumerate}
    \item Check your internet connection
    \item Use PyTorch's CDN (faster):
    \begin{lstlisting}[language=bash]
pip install torch torchvision torchaudio --index-url https://download.pytorch.org/whl/cpu
    \end{lstlisting}
    
    \item Install packages one at a time to identify the slow one
    
    \item Try a different mirror/index:
    \begin{lstlisting}[language=bash]
pip install -i https://pypi.tuna.tsinghua.edu.cn/simple torch
    \end{lstlisting}
\end{enumerate}
\end{solutionbox}

\subsection{Jupyter/Code Runs Very Slowly}

\begin{problembox}
Code execution is extremely slow, especially tensor operations.
\end{problembox}

\begin{solutionbox}
\begin{enumerate}
    \item Check if PyTorch is using CPU (expected without GPU):
    \begin{lstlisting}[language=Python]
import torch
print(torch.cuda.is_available())  # False is normal without NVIDIA GPU
    \end{lstlisting}
    
    \item Reduce batch sizes or data size
    
    \item For M1/M2/M3 Macs, make sure MPS is available:
    \begin{lstlisting}[language=Python]
print(torch.backends.mps.is_available())
    \end{lstlisting}
    
    \item Close other applications
    
    \item Check Task Manager/Activity Monitor for CPU/memory usage
\end{enumerate}
\end{solutionbox}

\section{Still Stuck?}

If you've tried everything and still have issues:

\begin{enumerate}
    \item \textbf{Document your error:}
    \begin{itemize}
        \item Copy the full error message
        \item Note your OS and Python version
        \item What you were trying to do
        \item What you've already tried
    \end{itemize}
    
    \item \textbf{Get help:}
    \begin{itemize}
        \item Post in course forum with your documentation
        \item Come to office hours
        \item Email instructor with details
    \end{itemize}
    
    \item \textbf{Temporary workaround:}
    \begin{itemize}
        \item Use GitHub Codespaces (see backup guide)
        \item Continue with exercises while we troubleshoot
        \item Get local setup working for next week
    \end{itemize}
\end{enumerate}

\section{Prevention Tips}

To avoid issues in the future:

\begin{enumerate}
    \item Always activate virtual environment before installing packages
    \item Keep virtual environment in project folder (easier to manage)
    \item Don't use \texttt{sudo} with pip
    \item Update pip regularly: \texttt{pip install --upgrade pip}
    \item Use \texttt{requirements.txt} to track dependencies
    \item Back up your environment:
    \begin{lstlisting}[language=bash]
pip freeze > requirements_backup.txt
    \end{lstlisting}
\end{enumerate}

\end{document}
