\documentclass[11pt,a4paper]{article}
\usepackage[margin=2.5cm]{geometry}
\usepackage{amsmath,amssymb}
\usepackage{enumitem}
\usepackage{hyperref}
\usepackage{listings}
\usepackage{xcolor}
\usepackage{tcolorbox}
\usepackage{fontawesome5}
\usepackage{graphicx}

% Code styling
\lstset{
    basicstyle=\ttfamily\small,
    keywordstyle=\color{blue},
    commentstyle=\color{gray},
    stringstyle=\color{red},
    showstringspaces=false,
    breaklines=true,
    frame=single,
    backgroundcolor=\color{gray!10}
}

% Custom boxes
\newtcolorbox{warningbox}{
    colback=orange!5!white,
    colframe=orange!75!black,
    title=\faExclamationTriangle\ Important
}

\newtcolorbox{infobox}{
    colback=blue!5!white,
    colframe=blue!75!black,
    title=\faInfoCircle\ Note
}

\newtcolorbox{tipbox}{
    colback=green!5!white,
    colframe=green!75!black,
    title=\faLightbulb\ Tip
}

\title{Introduction to Deep Learning\\
GitHub Codespaces - Backup Setup Guide}
\author{MSc Computer Science}
\date{Week 1}

\begin{document}

\maketitle

\tableofcontents
\newpage

\section{Introduction}

\begin{warningbox}
This guide is a \textbf{backup option} for students who are having trouble with local installation. You should still try to get your local environment working, as it's better for long-term development skills.
\end{warningbox}

\begin{infobox}
GitHub Codespaces provides a cloud-based development environment with VS Code in your browser. It's free for students with 60 hours/month of usage.
\end{infobox}

\subsection{When to Use Codespaces}

Use GitHub Codespaces if:
\begin{itemize}
    \item Your local Python installation is broken and you can't fix it quickly
    \item You're waiting for IT support to resolve installation issues
    \item You need to work from a different computer temporarily
    \item Your computer doesn't have enough disk space
\end{itemize}

\subsection{Advantages}

\begin{itemize}
    \item No local installation needed
    \item Works from any computer with a web browser
    \item Pre-configured environment
    \item Integrated with GitHub for easy code management
    \item VS Code interface (similar to local development)
    \item Free GPU access (limited)
\end{itemize}

\subsection{Limitations}

\begin{itemize}
    \item Requires internet connection
    \item 60 hours/month free quota (should be enough for this course)
    \item Session timeouts after 30 minutes of inactivity
    \item Files are saved in the cloud (make sure to commit/download)
\end{itemize}

\section{Prerequisites}

\subsection{GitHub Account}

You need a GitHub account:
\begin{enumerate}
    \item Go to \url{https://github.com/signup}
    \item Create a free account
    \item Verify your email address
\end{enumerate}

\subsection{GitHub Student Benefits (Recommended)}

Get free GitHub Pro with student benefits:
\begin{enumerate}
    \item Go to \url{https://education.github.com/pack}
    \item Click ``Sign up for Student Developer Pack''
    \item Verify your student status (using university email)
    \item Get increased Codespaces quota and other benefits
\end{enumerate}

\section{Setting Up Your Repository}

\subsection{Option A: Fork Course Repository (Recommended)}

If the instructor provides a course repository:

\begin{enumerate}
    \item Go to the course repository URL (provided by instructor)
    \item Click the ``Fork'' button (top right)
    \item This creates your own copy of the repository
    \item You now have your own repository: \texttt{github.com/YourUsername/deep-learning-course}
\end{enumerate}

\subsection{Option B: Create New Repository}

If starting from scratch:

\begin{enumerate}
    \item Go to \url{https://github.com/new}
    \item Repository name: \texttt{deep-learning-course}
    \item Description: ``Deep Learning Course - Week 1''
    \item Select ``Public'' or ``Private''
    \item Check ``Add a README file''
    \item Check ``Add .gitignore'' → select ``Python''
    \item Click ``Create repository''
\end{enumerate}

\section{Creating a Codespace}

\subsection{From Your Repository}

\begin{enumerate}
    \item Navigate to your repository on GitHub
    \item Click the green ``Code'' button
    \item Select the ``Codespaces'' tab
    \item Click ``Create codespace on main''
    \item Wait 1-2 minutes for the environment to build
\end{enumerate}

\begin{infobox}
The first time you create a Codespace, it may take a few minutes to set up. Subsequent launches will be faster.
\end{infobox}

\subsection{Codespace Interface}

Once loaded, you'll see:
\begin{itemize}
    \item Left sidebar: File explorer, Git, extensions
    \item Center: Code editor
    \item Bottom: Terminal
    \item Top: Menu bar
\end{itemize}

It looks and works like VS Code!

\section{Setting Up Python Environment}

\subsection{Check Python Version}

In the terminal (bottom panel), run:
\begin{lstlisting}[language=bash]
python3 --version
\end{lstlisting}

Should show Python 3.10 or later.

\subsection{Create Virtual Environment}

\begin{lstlisting}[language=bash]
# Create virtual environment
python3 -m venv deep_learning_env

# Activate it
source deep_learning_env/bin/activate

# You should see (deep_learning_env) in the prompt
\end{lstlisting}

\subsection{Install Required Packages}

\subsubsection{Using requirements.txt}

If you have the \texttt{requirements.txt} file:

\begin{enumerate}
    \item Upload \texttt{requirements.txt} to your repository:
    \begin{itemize}
        \item Drag and drop into file explorer, or
        \item Use terminal: create file and paste content
    \end{itemize}
    
    \item Install packages:
    \begin{lstlisting}[language=bash]
pip install --upgrade pip
pip install -r requirements.txt
    \end{lstlisting}
\end{enumerate}

\subsubsection{Manual Installation}

\begin{lstlisting}[language=bash]
pip install torch torchvision torchaudio
pip install jupyter notebook
pip install matplotlib numpy pandas
pip install ipywidgets
\end{lstlisting}

\begin{tipbox}
Installation in Codespaces is usually faster than local due to GitHub's fast network connection.
\end{tipbox}

\section{Using Jupyter Notebooks}

\subsection{Opening Notebooks}

\begin{enumerate}
    \item Upload your \texttt{.ipynb} file:
    \begin{itemize}
        \item Drag and drop into file explorer
        \item Or clone from another repository
    \end{itemize}
    
    \item Click on the \texttt{.ipynb} file to open it
    
    \item VS Code will render it as a Jupyter notebook
\end{enumerate}

\subsection{Selecting Kernel}

\begin{enumerate}
    \item Click ``Select Kernel'' button (top right of notebook)
    \item Choose ``Python Environments''
    \item Select your \texttt{deep\_learning\_env}
\end{enumerate}

\subsection{Running Cells}

\begin{itemize}
    \item Click the play button next to a cell, or
    \item Press Shift + Enter to run current cell and move to next
    \item Press Ctrl/Cmd + Enter to run current cell
\end{itemize}

\section{Working with Files}

\subsection{Uploading Files}

\begin{enumerate}
    \item Drag and drop files into the file explorer (left sidebar)
    \item Or use terminal:
    \begin{lstlisting}[language=bash]
# Download from URL
wget https://example.com/file.ipynb

# Or upload via GitHub web interface, then git pull
    \end{lstlisting}
\end{enumerate}

\subsection{Downloading Files}

\begin{enumerate}
    \item Right-click file in explorer → Download
    \item Or commit and push to GitHub, then download from there
\end{enumerate}

\subsection{Saving Your Work}

\begin{warningbox}
Codespaces can timeout! Make sure to save your work regularly.
\end{warningbox}

\textbf{Method 1: Git Commit (Recommended)}
\begin{lstlisting}[language=bash]
# Stage all changes
git add .

# Commit with message
git commit -m "Week 1 exercises completed"

# Push to GitHub
git push
\end{lstlisting}

\textbf{Method 2: Auto-save}
\begin{itemize}
    \item Files are auto-saved in Codespace
    \item But will be lost if Codespace is deleted
    \item Always commit important work!
\end{itemize}

\section{Managing Your Codespace}

\subsection{Stopping a Codespace}

When you're done working:
\begin{enumerate}
    \item Click ``Codespaces'' in bottom left corner
    \item Select ``Stop Current Codespace''
    \item Or just close the browser tab (auto-stops after 30 min)
\end{enumerate}

\begin{tipbox}
Stopping your Codespace saves your free quota hours!
\end{tipbox}

\subsection{Restarting a Codespace}

\begin{enumerate}
    \item Go to \url{https://github.com/codespaces}
    \item Find your codespace in the list
    \item Click the name to reopen it
    \item Remember to reactivate virtual environment:
    \begin{lstlisting}[language=bash]
source deep_learning_env/bin/activate
    \end{lstlisting}
\end{enumerate}

\subsection{Deleting a Codespace}

If you want to start fresh:
\begin{enumerate}
    \item Go to \url{https://github.com/codespaces}
    \item Click ``...'' next to your codespace
    \item Select ``Delete''
    \item Create a new one from your repository
\end{enumerate}

\section{Tips and Best Practices}

\subsection{Keyboard Shortcuts}

Learn these VS Code shortcuts:
\begin{itemize}
    \item Ctrl/Cmd + P: Quick file open
    \item Ctrl/Cmd + Shift + P: Command palette
    \item Ctrl/Cmd + `: Toggle terminal
    \item Ctrl/Cmd + B: Toggle sidebar
\end{itemize}

\subsection{Extensions}

Install helpful extensions:
\begin{enumerate}
    \item Click Extensions icon (left sidebar)
    \item Search and install:
    \begin{itemize}
        \item Python (Microsoft) - should be pre-installed
        \item Jupyter (Microsoft) - should be pre-installed
        \item Pylance - better Python IntelliSense
    \end{itemize}
\end{enumerate}

\subsection{Git Integration}

Use the built-in Git interface:
\begin{enumerate}
    \item Click Source Control icon (left sidebar)
    \item See changed files
    \item Stage changes (+ button)
    \item Write commit message
    \item Click checkmark to commit
    \item Click ``...'' → Push
\end{enumerate}

\subsection{Terminal Tips}

\begin{lstlisting}[language=bash]
# Create new terminal: Ctrl+Shift+`

# Split terminal: Click split icon

# Multiple terminals for different tasks:
# Terminal 1: Jupyter notebook
# Terminal 2: General commands
\end{lstlisting}

\section{Common Issues}

\subsection{Codespace Won't Start}

\begin{itemize}
    \item Check your internet connection
    \item Wait a few minutes and try again
    \item Check GitHub status: \url{https://www.githubstatus.com/}
    \item Try creating a new Codespace
\end{itemize}

\subsection{Out of Free Hours}

\begin{itemize}
    \item You get 60 hours/month free (120 with Student Pack)
    \item Check usage: \url{https://github.com/settings/billing}
    \item Stop Codespaces when not in use
    \item Use local environment when possible
\end{itemize}

\subsection{Kernel Died / Module Not Found}

\begin{enumerate}
    \item Make sure virtual environment is activated:
    \begin{lstlisting}[language=bash]
source deep_learning_env/bin/activate
    \end{lstlisting}
    
    \item Reinstall packages if needed
    
    \item Restart kernel: Click ``Restart'' button in notebook toolbar
\end{enumerate}

\subsection{Files Not Saving}

\begin{enumerate}
    \item Check file is not read-only
    \item Manually save: Ctrl/Cmd + S
    \item Check Codespace has not timed out
    \item Commit to Git to be safe
\end{enumerate}

\section{Transitioning to Local Setup}

Once your local installation is working:

\subsection{Downloading Your Work}

\begin{enumerate}
    \item Commit all changes in Codespace:
    \begin{lstlisting}[language=bash]
git add .
git commit -m "All my work"
git push
    \end{lstlisting}
    
    \item On your local machine:
    \begin{lstlisting}[language=bash]
git clone https://github.com/YourUsername/deep-learning-course
cd deep-learning-course
    \end{lstlisting}
    
    \item Set up local environment (see Setup Guide)
    
    \item Continue working locally
\end{enumerate}

\subsection{Keeping Codespace as Backup}

\begin{itemize}
    \item Keep your Codespace but stop it
    \item Work locally most of the time
    \item Use Codespace only when:
    \begin{itemize}
        \item Working from different computer
        \item Local environment breaks
        \item Need more compute power
    \end{itemize}
\end{itemize}

\section{Advanced: Using GPU (Optional)}

\begin{infobox}
GitHub Codespaces offers limited free GPU access. This is optional and not required for Week 1.
\end{infobox}

\subsection{Requesting GPU}

\begin{enumerate}
    \item When creating Codespace, click ``...'' → ``Configure dev container''
    \item Select machine type with GPU
    \item Note: This uses quota faster
\end{enumerate}

\subsection{Checking GPU}

\begin{lstlisting}[language=Python]
import torch
print(f"CUDA available: {torch.cuda.is_available()}")
if torch.cuda.is_available():
    print(f"GPU: {torch.cuda.get_device_name(0)}")
\end{lstlisting}

\section{Resources}

\begin{itemize}
    \item GitHub Codespaces Docs: \url{https://docs.github.com/en/codespaces}
    \item VS Code Tips: \url{https://code.visualstudio.com/docs/getstarted/tips-and-tricks}
    \item Git Tutorial: \url{https://git-scm.com/book/en/v2}
    \item Course Forum: [link provided by instructor]
\end{itemize}

\section{Getting Help}

If you have issues with Codespaces:

\begin{enumerate}
    \item Check GitHub Codespaces documentation
    \item Post in course forum
    \item Email instructor with:
    \begin{itemize}
        \item Screenshot of error
        \item What you were trying to do
        \item Your repository URL
    \end{itemize}
    \item Come to office hours
\end{enumerate}

\section{Summary}

\begin{enumerate}
    \item Create GitHub account
    \item Fork/create course repository
    \item Create Codespace
    \item Set up virtual environment
    \item Install packages
    \item Work in Jupyter notebooks
    \item \textbf{Commit and push regularly!}
    \item Stop Codespace when done
    \item Transition to local setup when possible
\end{enumerate}

\begin{warningbox}
Remember: Codespaces is a \textbf{temporary solution}. Work on getting your local environment set up for the long term!
\end{warningbox}

\end{document}
