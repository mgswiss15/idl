\documentclass[11pt,a4paper]{article}
\usepackage[margin=2.5cm]{geometry}
\usepackage{amsmath,amssymb}
\usepackage{enumitem}
\usepackage{hyperref}
\usepackage{listings}
\usepackage{xcolor}
\usepackage{tcolorbox}
\usepackage{fontawesome5}

% Code styling
\lstset{
    basicstyle=\ttfamily\small,
    keywordstyle=\color{blue},
    commentstyle=\color{gray},
    stringstyle=\color{red},
    showstringspaces=false,
    breaklines=true,
    frame=single,
    backgroundcolor=\color{gray!10}
}

% Custom boxes
\newtcolorbox{warningbox}{
    colback=red!5!white,
    colframe=red!75!black,
    title=\faExclamationTriangle\ Important
}

\newtcolorbox{infobox}{
    colback=blue!5!white,
    colframe=blue!75!black,
    title=\faInfoCircle\ Note
}

\newtcolorbox{tipbox}{
    colback=green!5!white,
    colframe=green!75!black,
    title=\faLightbulb\ Tip
}

\newtcolorbox{problembox}{
    colback=red!5!white,
    colframe=red!75!black,
    title=\faBug\ Problem
}

\newtcolorbox{solutionbox}{
    colback=green!5!white,
    colframe=green!75!black,
    title=\faCheck\ Solution
}

\title{Introduction to Deep Learning\\
Local Development Environment Setup Guide\\
\textbf{macOS}}
\author{MSc Artificial Intelligence}
\date{}

\begin{document}

\maketitle

\tableofcontents
\newpage

\section{Overview}

This guide will help you set up a local Python development environment on \textbf{macOS} for the Deep Learning course. You will:
\begin{itemize}
    \item Install Python 3.10 or later
    \item Create a virtual environment
    \item Install PyTorch and required packages
    \item Verify your installation with Jupyter Notebook
\end{itemize}

\begin{warningbox}
Please complete this setup \textbf{before} the first exercise session. If you encounter problems, consult the troubleshooting guide or use GitHub Codespaces as a temporary backup.
\end{warningbox}

\begin{infobox}
Estimated time: 30-45 minutes depending on your internet speed.\\
This guide covers both Intel and Apple Silicon (M1/M2/M3) Macs.
\end{infobox}

\section{Step 1: Check Existing Python Installation}

macOS usually comes with Python, but it's often an older version.

\subsection{Open Terminal}

Press \texttt{Cmd + Space}, type ``Terminal'', and press Enter.

\subsection{Check Python Version}

In terminal, run:
\begin{lstlisting}[language=bash]
python3 --version
\end{lstlisting}

\textbf{If you see \texttt{Python 3.10.x} or higher:}
\begin{itemize}
    \item Great! Skip to Step 3 (Setting Up Virtual Environment)
\end{itemize}

\textbf{If you see an older version or error:}
\begin{itemize}
    \item Continue to Step 2 to install Python
\end{itemize}

\begin{tipbox}
Don't use \texttt{python} (without the 3) - this often points to Python 2.7 on macOS.
\end{tipbox}

\section{Step 2: Install Python}

\subsection{Option 1: Using Homebrew (Recommended)}

Homebrew is the most popular package manager for macOS.

\subsubsection{Install Homebrew (if not already installed)}

Check if Homebrew is installed:
\begin{lstlisting}[language=bash]
brew --version
\end{lstlisting}

If not installed, install it:
\begin{lstlisting}[language=bash]
/bin/bash -c "$(curl -fsSL https://raw.githubusercontent.com/Homebrew/install/HEAD/install.sh)"
\end{lstlisting}

Follow the instructions to add Homebrew to your PATH.

\subsubsection{Install Python}

\begin{lstlisting}[language=bash]
brew install python@3.11
\end{lstlisting}

Wait 2-5 minutes for installation to complete.

\subsubsection{Verify Installation}

\begin{lstlisting}[language=bash]
python3 --version
\end{lstlisting}

Should show Python 3.11.x

\subsection{Option 2: Official Python Installer}

If you prefer not to use Homebrew:

\begin{enumerate}
    \item Go to \url{https://www.python.org/downloads/macos/}
    \item Click ``Download Python 3.11.x'' (latest version)
    \item Open the downloaded \texttt{.pkg} file
    \item Follow the installation wizard
    \item Use default options and click ``Install''
    \item Enter your password when prompted
    \item Click ``Close'' when finished
\end{enumerate}

\subsubsection{Install Certificates (Important!)}

After installing Python, you need to install SSL certificates:

\begin{lstlisting}[language=bash]
# Replace 3.11 with your Python version
/Applications/Python\ 3.11/Install\ Certificates.command
\end{lstlisting}

This prevents SSL errors when downloading packages.

\subsection{Verify Installation}

Open a new terminal and run:
\begin{lstlisting}[language=bash]
python3 --version
pip3 --version
\end{lstlisting}

Both should work without errors.

\section{Step 3: Install Xcode Command Line Tools}

These are required for some Python packages.

\subsection{Install Command Line Tools}

\begin{lstlisting}[language=bash]
xcode-select --install
\end{lstlisting}

\begin{itemize}
    \item Click ``Install'' in the popup window
    \item Accept the license agreement
    \item Wait 5-10 minutes for installation
\end{itemize}

\begin{infobox}
If you get ``command line tools are already installed'', that's fine - skip this step.
\end{infobox}

\section{Step 4: Create Project Folder}

\subsection{Navigate to Documents}

\begin{lstlisting}[language=bash]
cd ~/Documents
\end{lstlisting}

\subsection{Create and Navigate to Project Folder}

\begin{lstlisting}[language=bash]
mkdir DeepLearning
cd DeepLearning
\end{lstlisting}

\begin{tipbox}
You can verify your current location with:
\begin{lstlisting}[language=bash]
pwd
\end{lstlisting}
Should show: \texttt{/Users/yourusername/Documents/DeepLearning}
\end{tipbox}

\section{Step 5: Create Virtual Environment}

A virtual environment keeps your project dependencies isolated.

\subsection{Create the Virtual Environment}

In your \texttt{DeepLearning} folder:
\begin{lstlisting}[language=bash]
python3 -m venv deep_learning_env
\end{lstlisting}

This creates a new folder called \texttt{deep\_learning\_env} containing:
\begin{itemize}
    \item Python interpreter
    \item pip package manager
    \item Space for installed packages
\end{itemize}

Wait 10-30 seconds for creation to complete.

\subsection{Activate the Virtual Environment}

\begin{lstlisting}[language=bash]
source deep_learning_env/bin/activate
\end{lstlisting}

\subsection{Verify Activation}

After activation, you should see:
\begin{lstlisting}
(deep_learning_env) computername:DeepLearning username$
\end{lstlisting}

The \texttt{(deep\_learning\_env)} prefix indicates the virtual environment is active!

\begin{tipbox}
To deactivate later, simply type:
\begin{lstlisting}[language=bash]
deactivate
\end{lstlisting}
\end{tipbox}

\begin{warningbox}
\textbf{Important:} You must use \texttt{source} to activate. Just running \texttt{deep\_learning\_env/bin/activate} without \texttt{source} will not work!
\end{warningbox}

\section{Step 6: Install Required Packages}

\begin{warningbox}
Make sure your virtual environment is activated! You should see \texttt{(deep\_learning\_env)} in your prompt.
\end{warningbox}

\subsection{Upgrade pip}

First, upgrade pip to the latest version:
\begin{lstlisting}[language=bash]
python3 -m pip install --upgrade pip
\end{lstlisting}

\subsection{Download requirements.txt}

\begin{enumerate}
    \item Download \texttt{requirements.txt} from the course repository
    \item Save it to your \texttt{DeepLearning} folder
    \item Verify it's there:
    \begin{lstlisting}[language=bash]
ls requirements.txt
    \end{lstlisting}
\end{enumerate}

\begin{tipbox}
You can download directly from terminal if you have the URL:
\begin{lstlisting}[language=bash]
curl -O https://url-to-requirements.txt
\end{lstlisting}
\end{tipbox}

\subsection{Install All Packages}

\begin{lstlisting}[language=bash]
pip install -r requirements.txt
\end{lstlisting}

\begin{infobox}
This will take 5-10 minutes as PyTorch is a large package (~700MB). You'll see progress bars for each package being downloaded and installed.
\end{infobox}

\subsection{Alternative: Manual Installation}

If you don't have \texttt{requirements.txt}, install packages individually:

\begin{lstlisting}[language=bash]
pip install torch torchvision torchaudio
pip install jupyter notebook
pip install matplotlib numpy pandas
pip install ipywidgets
\end{lstlisting}

\begin{infobox}
\textbf{For Apple Silicon Macs (M1/M2/M3):} PyTorch will automatically install the ARM64 version optimized for your processor. No special steps needed!
\end{infobox}

\section{Step 7: Verify Installation}

\subsection{Test Python and PyTorch}

\begin{enumerate}
    \item Start Python interpreter:
    \begin{lstlisting}[language=bash]
python3
    \end{lstlisting}
    
    \item In the Python prompt (\texttt{>>>}), run:
    \begin{lstlisting}[language=Python]
import torch
import platform

print(f"PyTorch version: {torch.__version__}")
print(f"Python architecture: {platform.machine()}")
print(f"CUDA available: {torch.cuda.is_available()}")
print(f"MPS available: {torch.backends.mps.is_available()}")

# Create a test tensor
x = torch.tensor([1, 2, 3])
print(f"Test tensor: {x}")
    \end{lstlisting}
    
    \item You should see output similar to:
    \begin{lstlisting}
PyTorch version: 2.x.x
Python architecture: arm64  # or x86_64 for Intel Macs
CUDA available: False
MPS available: True  # or False for Intel Macs
Test tensor: tensor([1, 2, 3])
    \end{lstlisting}
    
    \item Exit Python:
    \begin{lstlisting}[language=Python]
exit()
    \end{lstlisting}
\end{enumerate}

\begin{infobox}
\textbf{For M1/M2/M3 Macs:} \texttt{MPS available: True} means you can use Apple's Metal Performance Shaders for GPU acceleration (optional).

\textbf{For Intel Macs:} Both CUDA and MPS will be False - you'll use CPU, which is fine for this course.
\end{infobox}

\subsection{Launch Jupyter Notebook}

\begin{enumerate}
    \item Make sure your virtual environment is still activated
    
    \item Run:
    \begin{lstlisting}[language=bash]
jupyter notebook
    \end{lstlisting}
    
    \item A browser window should open automatically showing the Jupyter interface
    
    \item You should see your \texttt{DeepLearning} folder contents
\end{enumerate}

\begin{tipbox}
If the browser doesn't open automatically:
\begin{itemize}
    \item Look for a URL in the terminal output
    \item It will look like: \texttt{http://localhost:8888/?token=...}
    \item Copy this URL and paste it into your browser
\end{itemize}
\end{tipbox}

\subsection{Create and Test a Notebook}

\begin{enumerate}
    \item In Jupyter, click ``New'' → ``Python 3 (ipykernel)''
    
    \item In the first cell, type:
    \begin{lstlisting}[language=Python]
import torch
import matplotlib.pyplot as plt
import numpy as np

print("Setup successful!")
print(f"PyTorch version: {torch.__version__}")

# Simple test
x = torch.randn(5)
print(f"Random tensor: {x}")
    \end{lstlisting}
    
    \item Press \texttt{Shift + Enter} to run the cell
    
    \item If you see ``Setup successful!'' and no errors, everything is working!
    
    \item Close the notebook (File → Close and Halt)
    
    \item Stop Jupyter by pressing \texttt{Ctrl + C} twice in the terminal
\end{enumerate}

\section{Step 8: Daily Workflow}

Every time you want to work on the course:

\subsection{Starting Your Work Session}

\begin{enumerate}
    \item Open Terminal (\texttt{Cmd + Space} → ``Terminal'')
    
    \item Navigate to your project folder:
    \begin{lstlisting}[language=bash]
cd ~/Documents/DeepLearning
    \end{lstlisting}
    
    \item Activate virtual environment:
    \begin{lstlisting}[language=bash]
source deep_learning_env/bin/activate
    \end{lstlisting}
    
    \item Start Jupyter:
    \begin{lstlisting}[language=bash]
jupyter notebook
    \end{lstlisting}
    
    \item Work in your notebooks
\end{enumerate}

\subsection{Ending Your Work Session}

\begin{enumerate}
    \item Save your notebooks (\texttt{Cmd + S})
    
    \item Close notebooks in Jupyter (File → Close and Halt)
    
    \item Stop Jupyter: \texttt{Ctrl + C} twice in terminal
    
    \item Deactivate virtual environment:
    \begin{lstlisting}[language=bash]
deactivate
    \end{lstlisting}
    
    \item Close terminal (\texttt{Cmd + Q})
\end{enumerate}

\begin{tipbox}
Create a shell alias to make activation easier:
\begin{lstlisting}[language=bash]
# Add to ~/.zshrc (or ~/.bash_profile for older macOS)
alias dlenv='cd ~/Documents/DeepLearning && source deep_learning_env/bin/activate'
\end{lstlisting}
Then you can just type \texttt{dlenv} to navigate and activate!

To reload your shell config: \texttt{source ~/.zshrc}
\end{tipbox}

\section{Optional: IDE Setup}

While Jupyter Notebooks are great for exercises, you may want a full IDE for projects.

\subsection{VS Code (Recommended)}

\subsubsection{Installation}

\begin{enumerate}
    \item Go to \url{https://code.visualstudio.com/}
    \item Click ``Download Mac Universal'' (works for both Intel and Apple Silicon)
    \item Open the downloaded \texttt{.zip} file
    \item Drag \texttt{Visual Studio Code.app} to your Applications folder
    \item Launch VS Code from Applications
\end{enumerate}

\subsubsection{Setup for Python}

\begin{enumerate}
    \item Click Extensions icon (left sidebar) or press \texttt{Cmd + Shift + X}
    
    \item Search for and install:
    \begin{itemize}
        \item ``Python'' by Microsoft
        \item ``Jupyter'' by Microsoft
    \end{itemize}
    
    \item Open your DeepLearning folder: File → Open Folder
    
    \item Select Python interpreter:
    \begin{itemize}
        \item Press \texttt{Cmd + Shift + P}
        \item Type ``Python: Select Interpreter''
        \item Choose the one in \texttt{deep\_learning\_env/bin/python}
    \end{itemize}
    
    \item You can now open and run .ipynb files directly in VS Code!
\end{enumerate}

\subsection{PyCharm Community (Alternative)}

\begin{enumerate}
    \item Go to \url{https://www.jetbrains.com/pycharm/download/#section=mac}
    \item Download Community Edition (free)
    \item Open the \texttt{.dmg} file
    \item Drag PyCharm to Applications
    \item Launch PyCharm
    \item Open your DeepLearning folder
    \item Configure interpreter:
    \begin{itemize}
        \item PyCharm → Settings → Project → Python Interpreter
        \item Click gear icon → Add
        \item Choose ``Existing environment''
        \item Navigate to: \texttt{~/Documents/DeepLearning/deep\_learning\_env/bin/python}
    \end{itemize}
\end{enumerate}

\section{Optional: Apple Silicon GPU Support (M1/M2/M3)}

\begin{infobox}
Apple Silicon Macs can use Metal Performance Shaders (MPS) for GPU acceleration. This is optional and not required for Week 1, but can speed up training from Week 4 onwards.
\end{infobox}

\subsection{Check MPS Availability}

MPS should already be available if you have an M1/M2/M3 Mac:

\begin{lstlisting}[language=Python]
import torch

print(f"MPS available: {torch.backends.mps.is_available()}")
print(f"MPS built: {torch.backends.mps.is_built()}")
\end{lstlisting}

Should print:
\begin{lstlisting}
MPS available: True
MPS built: True
\end{lstlisting}

\subsection{Using MPS in Your Code}

\begin{lstlisting}[language=Python]
import torch

# Set device to MPS if available, otherwise CPU
device = torch.device("mps" if torch.backends.mps.is_available() else "cpu")
print(f"Using device: {device}")

# Create tensor on MPS
x = torch.randn(3, 3).to(device)
print(f"Tensor device: {x.device}")

# Or create directly on MPS
y = torch.randn(3, 3, device=device)
\end{lstlisting}

\begin{infobox}
You'll learn more about device management in later weeks. For now, just knowing it's available is enough!
\end{infobox}

\subsection{MPS Limitations}

Some operations may not be supported on MPS yet. If you encounter errors, fall back to CPU:

\begin{lstlisting}[language=Python]
try:
    x = x.to('mps')
except:
    print("Operation not supported on MPS, using CPU")
    x = x.to('cpu')
\end{lstlisting}

\section{Troubleshooting: Architecture Issues (M1/M2/M3)}

\subsection{Check Python Architecture}

Make sure you're using ARM64 Python, not x86\_64 (Rosetta):

\begin{lstlisting}[language=bash]
python3 -c "import platform; print(platform.machine())"
\end{lstlisting}

Should print: \texttt{arm64}

If it prints \texttt{x86\_64}, you installed x86 Python. Reinstall using Homebrew:

\begin{lstlisting}[language=bash]
# Uninstall old Python
brew uninstall python@3.11

# Make sure Homebrew is ARM64
arch -arm64 brew install python@3.11
\end{lstlisting}

\subsection{Rosetta 2}

Some packages may still require Rosetta 2. If prompted, install it:

\begin{lstlisting}[language=bash]
softwareupdate --install-rosetta
\end{lstlisting}

\section{Troubleshooting Common Issues}

\subsection{Python Not Found}

\begin{problembox}
Error: \texttt{python3: command not found}
\end{problembox}

\begin{solutionbox}
\begin{enumerate}
    \item Install Python (see Step 2)
    \item Check if Homebrew installed it:
    \begin{lstlisting}[language=bash]
which python3
/usr/local/bin/python3 --version
    \end{lstlisting}
    \item Try with version number:
    \begin{lstlisting}[language=bash]
python3.11 --version
    \end{lstlisting}
\end{enumerate}
\end{solutionbox}

\subsection{xcrun Error}

\begin{problembox}
Error: \texttt{xcrun: error: invalid active developer path}
\end{problembox}

\begin{solutionbox}
Install Xcode Command Line Tools:
\begin{lstlisting}[language=bash]
xcode-select --install
\end{lstlisting}

Click ``Install'' and wait for completion.
\end{solutionbox}

\subsection{SSL Certificate Errors}

\begin{problembox}
\texttt{SSL: CERTIFICATE\_VERIFY\_FAILED} when installing packages.
\end{problembox}

\begin{solutionbox}
Run the certificate installer that comes with Python:
\begin{lstlisting}[language=bash]
# Replace 3.11 with your version
/Applications/Python\ 3.11/Install\ Certificates.command
\end{lstlisting}

Or if installed via Homebrew:
\begin{lstlisting}[language=bash]
pip install --upgrade certifi
\end{lstlisting}
\end{solutionbox}

\subsection{Virtual Environment Won't Activate}

\begin{problembox}
After running activation, no \texttt{(deep\_learning\_env)} prefix appears.
\end{problembox}

\begin{solutionbox}
\begin{enumerate}
    \item Make sure you use \texttt{source}:
    \begin{lstlisting}[language=bash]
source deep_learning_env/bin/activate
    \end{lstlisting}
    
    \item Check if venv was created:
    \begin{lstlisting}[language=bash]
ls deep_learning_env/bin/
    \end{lstlisting}
    Should see \texttt{activate} file
    
    \item Try recreating:
    \begin{lstlisting}[language=bash]
rm -rf deep_learning_env
python3 -m venv deep_learning_env
source deep_learning_env/bin/activate
    \end{lstlisting}
\end{enumerate}
\end{solutionbox}

\subsection{Architecture Issues (Apple Silicon)}

\begin{problembox}
Package installation fails with architecture errors on M1/M2/M3 Mac.
\end{problembox}

\begin{solutionbox}
\textbf{Check Python architecture:}
\begin{lstlisting}[language=bash]
python3 -c "import platform; print(platform.machine())"
\end{lstlisting}

Should print \texttt{arm64}, not \texttt{x86\_64}.

\textbf{If showing x86\_64:}
\begin{enumerate}
    \item You have x86 Python (running under Rosetta)
    \item Uninstall and reinstall ARM64 version:
    \begin{lstlisting}[language=bash]
# If using Homebrew
brew uninstall python@3.11
arch -arm64 brew install python@3.11
    \end{lstlisting}
    
    \item Or download official ARM64 installer from python.org
\end{enumerate}

\textbf{Install Rosetta 2 if needed:}
\begin{lstlisting}[language=bash]
softwareupdate --install-rosetta
\end{lstlisting}
\end{solutionbox}

\subsection{Homebrew Issues}

\begin{problembox}
Homebrew commands not working or packages not found.
\end{problembox}

\begin{solutionbox}
\textbf{Add Homebrew to PATH:}

For Apple Silicon (M1/M2/M3):
\begin{lstlisting}[language=bash]
echo 'eval "$(/opt/homebrew/bin/brew shellenv)"' >> ~/.zshrc
source ~/.zshrc
\end{lstlisting}

For Intel Macs:
\begin{lstlisting}[language=bash]
echo 'eval "$(/usr/local/bin/brew shellenv)"' >> ~/.zshrc
source ~/.zshrc
\end{lstlisting}

\textbf{Update Homebrew:}
\begin{lstlisting}[language=bash]
brew update
brew upgrade
\end{lstlisting}
\end{solutionbox}

\subsection{Package Installation Fails}

\begin{problembox}
\texttt{pip install} fails with timeout or connection errors.
\end{problembox}

\begin{solutionbox}
\textbf{Try PyTorch CDN (faster):}
\begin{lstlisting}[language=bash]
pip install torch torchvision torchaudio --index-url https://download.pytorch.org/whl/cpu
\end{lstlisting}

\textbf{Increase timeout:}
\begin{lstlisting}[language=bash]
pip install --default-timeout=1000 torch
\end{lstlisting}

\textbf{Check connection:}
\begin{lstlisting}[language=bash]
ping pypi.org
\end{lstlisting}
\end{solutionbox}

\subsection{Jupyter Notebook Won't Start}

\begin{problembox}
\texttt{jupyter notebook} fails or browser doesn't open.
\end{problembox}

\begin{solutionbox}
\begin{enumerate}
    \item Verify Jupyter is installed:
    \begin{lstlisting}[language=bash]
pip install jupyter notebook
    \end{lstlisting}
    
    \item Try different port:
    \begin{lstlisting}[language=bash]
jupyter notebook --port=8889
    \end{lstlisting}
    
    \item Check if port 8888 is in use:
    \begin{lstlisting}[language=bash]
lsof -i :8888
    \end{lstlisting}
    
    \item Manually copy URL from terminal to browser
\end{enumerate}
\end{solutionbox}

\subsection{Module Not Found in Jupyter}

\begin{problembox}
\texttt{ModuleNotFoundError: No module named 'torch'} in Jupyter.
\end{problembox}

\begin{solutionbox}
Jupyter is using wrong Python environment!

\begin{enumerate}
    \item Activate virtual environment BEFORE starting Jupyter
    
    \item Check Python path in notebook:
    \begin{lstlisting}[language=Python]
import sys
print(sys.executable)
    \end{lstlisting}
    Should point to \texttt{deep\_learning\_env}
    
    \item Register environment as kernel:
    \begin{lstlisting}[language=bash]
python -m ipykernel install --user --name=deep_learning_env
    \end{lstlisting}
    Then: Kernel → Change kernel → deep\_learning\_env
\end{enumerate}
\end{solutionbox}

\subsection{MPS Not Working (Apple Silicon)}

\begin{problembox}
\texttt{torch.backends.mps.is\_available()} returns False on M1/M2/M3 Mac.
\end{problembox}

\begin{solutionbox}
\begin{enumerate}
    \item Make sure you have ARM64 Python (not x86\_64)
    \item Check macOS version (need macOS 12.3+):
    \begin{lstlisting}[language=bash]
sw_vers
    \end{lstlisting}
    \item Update PyTorch to latest version:
    \begin{lstlisting}[language=bash]
pip install --upgrade torch torchvision torchaudio
    \end{lstlisting}
    \item Some operations may not be MPS-compatible yet - use CPU as fallback
\end{enumerate}
\end{solutionbox}

\subsection{Permission Issues}

\begin{problembox}
Getting permission errors when creating files or installing packages.
\end{problembox}

\begin{solutionbox}
\begin{enumerate}
    \item Make sure you're working in your home directory (\texttt{\~{}/Documents})
    \item Never use \texttt{sudo} with pip
    \item If you used sudo accidentally, fix ownership:
    \begin{lstlisting}[language=bash]
sudo chown -R $USER:staff ~/Documents/DeepLearning
    \end{lstlisting}
\end{enumerate}
\end{solutionbox}

\section{Getting Help}

If you've tried the troubleshooting steps and still have issues:

\begin{enumerate}
    \item Document your error:
    \begin{itemize}
        \item Copy full error message
        \item Note your macOS version (\texttt{sw\_vers})
        \item Note your Mac type (Intel or M1/M2/M3)
        \item Note Python version and architecture
        \item What you were trying to do
        \item What you've already tried
    \end{itemize}
    
    \item Get help:
    \begin{itemize}
        \item Post in course forum with documentation
        \item Email instructor with details
        \item Come to office hours
    \end{itemize}
    
    \item Temporary workaround:
    \begin{itemize}
        \item Use GitHub Codespaces (see separate guide)
        \item Continue with exercises while troubleshooting local setup
    \end{itemize}
\end{enumerate}

\section{Next Steps}

\begin{enumerate}
    \item Download Week 1 exercise notebooks from course repository
    \item Place them in your \texttt{DeepLearning} folder
    \item Activate virtual environment
    \item Start Jupyter Notebook
    \item Open \texttt{week1\_exercises\_starter.ipynb}
    \item You're ready to start coding!
\end{enumerate}

\begin{tipbox}
\textbf{Create a habit:} Always activate your virtual environment before working on course materials!
\end{tipbox}

\section{Quick Reference}

\subsection{Essential Commands}

\begin{lstlisting}[language=bash]
# Navigate to project
cd ~/Documents/DeepLearning

# Activate virtual environment
source deep_learning_env/bin/activate

# Start Jupyter
jupyter notebook

# Stop Jupyter
# Press Ctrl+C twice

# Deactivate virtual environment
deactivate
\end{lstlisting}

\subsection{Troubleshooting Quick Fixes}

\begin{itemize}
    \item \textbf{python3: command not found}: Install Python (Step 2)
    \item \textbf{SSL certificate errors}: Run Install Certificates.command
    \item \textbf{xcrun: error}: Install Xcode Command Line Tools (Step 3)
    \item \textbf{Virtual env won't activate}: Use \texttt{source} command
    \item \textbf{Module not found in Jupyter}: Activate venv before starting Jupyter
    \item \textbf{Architecture is x86\_64 on M1/M2/M3}: Reinstall ARM64 Python
\end{itemize}

\subsection{Default Shell Note}

\begin{infobox}
macOS Catalina (10.15) and later use \texttt{zsh} as the default shell. Configuration file: \texttt{~/.zshrc}

Older versions use \texttt{bash}. Configuration file: \texttt{~/.bash\_profile}
\end{infobox}

\end{document}
