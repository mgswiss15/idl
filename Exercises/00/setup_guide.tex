\documentclass[11pt,a4paper]{article}
\usepackage[margin=2.5cm]{geometry}
\usepackage{amsmath,amssymb}
\usepackage{enumitem}
\usepackage{hyperref}
\usepackage{listings}
\usepackage{xcolor}
\usepackage{tcolorbox}
\usepackage{fontawesome5}

% Code styling
\lstset{
    basicstyle=\ttfamily\small,
    keywordstyle=\color{blue},
    commentstyle=\color{gray},
    stringstyle=\color{red},
    showstringspaces=false,
    breaklines=true,
    frame=single,
    backgroundcolor=\color{gray!10}
}

% Custom boxes
\newtcolorbox{warningbox}{
    colback=red!5!white,
    colframe=red!75!black,
    title=\faExclamationTriangle\ Important
}

\newtcolorbox{infobox}{
    colback=blue!5!white,
    colframe=blue!75!black,
    title=\faInfoCircle\ Note
}

\newtcolorbox{tipbox}{
    colback=green!5!white,
    colframe=green!75!black,
    title=\faLightbulb\ Tip
}

\title{Introduction to Deep Learning\\
Local Development Environment Setup Guide}
\author{MSc Computer Science}
\date{Week 1}

\begin{document}

\maketitle

\tableofcontents
\newpage

\section{Overview}

This guide will help you set up a local Python development environment for the Deep Learning course. You will:
\begin{itemize}
    \item Install Python 3.10 or later
    \item Create a virtual environment
    \item Install PyTorch and required packages
    \item Verify your installation with Jupyter Notebook
\end{itemize}

\begin{warningbox}
Please complete this setup \textbf{before} the first exercise session. If you encounter problems, consult the troubleshooting guide or use GitHub Codespaces as a temporary backup.
\end{warningbox}

\begin{infobox}
Estimated time: 30-45 minutes depending on your internet speed and operating system.
\end{infobox}

\section{Prerequisites}

\subsection{Check Python Installation}

First, check if you have Python 3.10 or later installed:

\subsubsection{Windows}
Open Command Prompt (Win + R, type \texttt{cmd}, press Enter) and run:
\begin{lstlisting}[language=bash]
python --version
\end{lstlisting}

\subsubsection{Linux}
Open Terminal (Ctrl + Alt + T) and run:
\begin{lstlisting}[language=bash]
python3 --version
\end{lstlisting}

\subsubsection{macOS}
Open Terminal (Cmd + Space, type ``Terminal'', press Enter) and run:
\begin{lstlisting}[language=bash]
python3 --version
\end{lstlisting}

\vskip 1em
If you see \texttt{Python 3.10.x} or higher, you're good to go! Otherwise, proceed to Section \ref{sec:install_python}.

\section{Installing Python}
\label{sec:install_python}

\subsection{Windows}

\begin{enumerate}
    \item Go to \url{https://www.python.org/downloads/}
    \item Download Python 3.11 or later (the latest stable version)
    \item Run the installer
    \item \textbf{IMPORTANT:} Check ``Add Python to PATH'' before clicking Install
    \item Click ``Install Now''
    \item Verify installation by opening a new Command Prompt and running:
    \begin{lstlisting}[language=bash]
python --version
    \end{lstlisting}
\end{enumerate}

\begin{warningbox}
Make sure to check ``Add Python to PATH''! Without this, Python commands won't work from the command line.
\end{warningbox}

\subsection{Linux (Ubuntu/Debian)}

Python 3 is usually pre-installed. If not, or if you need a newer version:

\begin{lstlisting}[language=bash]
# Update package list
sudo apt update

# Install Python 3.11
sudo apt install python3.11 python3.11-venv python3-pip

# Verify installation
python3.11 --version
\end{lstlisting}

For other distributions, use your package manager (\texttt{dnf}, \texttt{pacman}, etc.).

\subsection{macOS}

\subsubsection{Option 1: Using Homebrew (Recommended)}
If you have Homebrew installed:
\begin{lstlisting}[language=bash]
# Install Python
brew install python@3.11

# Verify installation
python3 --version
\end{lstlisting}

\subsubsection{Option 2: Official Installer}
\begin{enumerate}
    \item Go to \url{https://www.python.org/downloads/macos/}
    \item Download the macOS installer for Python 3.11 or later
    \item Run the installer package
    \item Verify installation in Terminal:
    \begin{lstlisting}[language=bash]
python3 --version
    \end{lstlisting}
\end{enumerate}

\section{Setting Up Virtual Environment}

Virtual environments keep your project dependencies isolated and manageable.

\subsection{Windows}

\begin{enumerate}
    \item Open Command Prompt
    \item Navigate to where you want to create your project folder:
    \begin{lstlisting}[language=bash]
cd C:\Users\YourUsername\Documents
mkdir DeepLearning
cd DeepLearning
    \end{lstlisting}
    
    \item Create virtual environment:
    \begin{lstlisting}[language=bash]
python -m venv deep_learning_env
    \end{lstlisting}
    
    \item Activate the virtual environment:
    \begin{lstlisting}[language=bash]
deep_learning_env\Scripts\activate
    \end{lstlisting}
    
    \item You should see \texttt{(deep\_learning\_env)} at the beginning of your command prompt
\end{enumerate}

\begin{tipbox}
To deactivate the virtual environment later, simply type \texttt{deactivate}.
\end{tipbox}

\subsection{Linux}

\begin{enumerate}
    \item Open Terminal
    \item Navigate to your preferred location:
    \begin{lstlisting}[language=bash]
cd ~/Documents
mkdir DeepLearning
cd DeepLearning
    \end{lstlisting}
    
    \item Create virtual environment:
    \begin{lstlisting}[language=bash]
python3 -m venv deep_learning_env
    \end{lstlisting}
    
    \item Activate the virtual environment:
    \begin{lstlisting}[language=bash]
source deep_learning_env/bin/activate
    \end{lstlisting}
    
    \item You should see \texttt{(deep\_learning\_env)} at the beginning of your prompt
\end{enumerate}

\subsection{macOS}

\begin{enumerate}
    \item Open Terminal
    \item Navigate to your preferred location:
    \begin{lstlisting}[language=bash]
cd ~/Documents
mkdir DeepLearning
cd DeepLearning
    \end{lstlisting}
    
    \item Create virtual environment:
    \begin{lstlisting}[language=bash]
python3 -m venv deep_learning_env
    \end{lstlisting}
    
    \item Activate the virtual environment:
    \begin{lstlisting}[language=bash]
source deep_learning_env/bin/activate
    \end{lstlisting}
    
    \item You should see \texttt{(deep\_learning\_env)} at the beginning of your prompt
\end{enumerate}

\section{Installing Required Packages}

\begin{warningbox}
Make sure your virtual environment is activated before installing packages! You should see \texttt{(deep\_learning\_env)} in your prompt.
\end{warningbox}

\subsection{Download requirements.txt}

Download the \texttt{requirements.txt} file from the course repository or create it yourself (see Section \ref{sec:requirements}).

\subsection{Install Packages}

With your virtual environment activated, run:

\subsubsection{Windows}
\begin{lstlisting}[language=bash]
python -m pip install --upgrade pip
pip install -r requirements.txt
\end{lstlisting}

\subsubsection{Linux/macOS}
\begin{lstlisting}[language=bash]
python3 -m pip install --upgrade pip
pip install -r requirements.txt
\end{lstlisting}

\begin{infobox}
Installation may take 5-10 minutes depending on your internet connection. PyTorch is a large package (~700MB).
\end{infobox}

\subsection{Manual Installation (if requirements.txt not available)}

\begin{lstlisting}[language=bash]
pip install torch torchvision torchaudio
pip install jupyter notebook
pip install matplotlib numpy
pip install ipywidgets
\end{lstlisting}

\section{Verifying Installation}

\subsection{Check PyTorch Installation}

Run Python in your terminal:
\begin{lstlisting}[language=bash]
# Windows:
python

# Linux/macOS:
python3
\end{lstlisting}

Then in the Python interpreter:
\begin{lstlisting}[language=Python]
import torch
print(f"PyTorch version: {torch.__version__}")
print(f"CUDA available: {torch.cuda.is_available()}")

# Create a simple tensor
x = torch.tensor([1, 2, 3])
print(f"Test tensor: {x}")
\end{lstlisting}

You should see the PyTorch version and a tensor printed. Type \texttt{exit()} to leave Python.

\begin{infobox}
\texttt{CUDA available: False} is normal if you don't have an NVIDIA GPU. PyTorch will use CPU, which is fine for this course.
\end{infobox}

\subsection{Launch Jupyter Notebook}

With your virtual environment activated:

\begin{lstlisting}[language=bash]
jupyter notebook
\end{lstlisting}

This should open a web browser with the Jupyter interface. You should see your \texttt{DeepLearning} folder.

\begin{tipbox}
If the browser doesn't open automatically, look for a URL in the terminal output (usually \texttt{http://localhost:8888/...}) and copy it into your browser.
\end{tipbox}

\subsection{Test Notebook}

\begin{enumerate}
    \item In Jupyter, click ``New'' → ``Python 3 (ipykernel)''
    \item In the first cell, type:
    \begin{lstlisting}[language=Python]
import torch
import matplotlib.pyplot as plt
import numpy as np

print("Setup successful!")
print(f"PyTorch version: {torch.__version__}")
    \end{lstlisting}
    
    \item Press Shift + Enter to run the cell
    \item If you see ``Setup successful!'' and the PyTorch version, everything is working!
\end{enumerate}

\section{Daily Workflow}

Every time you want to work on the course:

\subsection{Windows}
\begin{lstlisting}[language=bash]
# Navigate to your project folder
cd C:\Users\YourUsername\Documents\DeepLearning

# Activate virtual environment
deep_learning_env\Scripts\activate

# Start Jupyter
jupyter notebook

# When done, deactivate
deactivate
\end{lstlisting}

\subsection{Linux/macOS}
\begin{lstlisting}[language=bash]
# Navigate to your project folder
cd ~/Documents/DeepLearning

# Activate virtual environment
source deep_learning_env/bin/activate

# Start Jupyter
jupyter notebook

# When done, deactivate
deactivate
\end{lstlisting}

\section{IDE Setup (Optional but Recommended)}

While Jupyter Notebooks are great for exercises, you may want a full IDE for projects:

\subsection{VS Code (Recommended)}

\begin{enumerate}
    \item Download from \url{https://code.visualstudio.com/}
    \item Install the Python extension (Microsoft)
    \item Install the Jupyter extension (Microsoft)
    \item Open your project folder
    \item Select your virtual environment:
    \begin{itemize}
        \item Press Ctrl/Cmd + Shift + P
        \item Type ``Python: Select Interpreter''
        \item Choose the one in \texttt{deep\_learning\_env}
    \end{itemize}
\end{enumerate}

\subsection{PyCharm (Alternative)}

\begin{enumerate}
    \item Download PyCharm Community Edition from \url{https://www.jetbrains.com/pycharm/}
    \item Open your project folder
    \item Configure interpreter: File → Settings → Project → Python Interpreter
    \item Select your \texttt{deep\_learning\_env}
\end{enumerate}

\section{GPU Support (Optional)}

\begin{infobox}
GPU support is \textbf{not required} for this course, but can speed up training significantly from Week 4 onwards.
\end{infobox}

\subsection{Check if you have NVIDIA GPU}

\subsubsection{Windows}
\begin{enumerate}
    \item Right-click on desktop → Display settings → Advanced display
    \item Or: Open Task Manager → Performance tab
\end{enumerate}

\subsubsection{Linux}
\begin{lstlisting}[language=bash]
lspci | grep -i nvidia
\end{lstlisting}

\subsubsection{macOS}
Recent Macs use Apple Silicon (M1/M2/M3), which has different support. For Intel Macs, check: About This Mac → Graphics.

\subsection{Installing CUDA Support (NVIDIA GPU only)}

\begin{enumerate}
    \item Install NVIDIA GPU drivers from \url{https://www.nvidia.com/drivers}
    \item Install CUDA Toolkit 11.8 or 12.1 from \url{https://developer.nvidia.com/cuda-downloads}
    \item Reinstall PyTorch with CUDA support:
    \begin{lstlisting}[language=bash]
# Uninstall current PyTorch
pip uninstall torch torchvision torchaudio

# Install with CUDA 11.8 (check PyTorch website for latest)
pip install torch torchvision torchaudio --index-url https://download.pytorch.org/whl/cu118
    \end{lstlisting}
    
    \item Verify:
    \begin{lstlisting}[language=Python]
import torch
print(torch.cuda.is_available())  # Should print True
print(torch.cuda.get_device_name(0))  # Your GPU name
    \end{lstlisting}
\end{enumerate}

\begin{warningbox}
GPU setup can be tricky. If you have issues, stick with CPU for now. We can revisit GPU setup in Week 4 when it becomes more beneficial.
\end{warningbox}

\subsection{Apple Silicon (M1/M2/M3) Support}

PyTorch supports Apple's Metal Performance Shaders (MPS):

\begin{lstlisting}[language=Python]
import torch

# Check MPS availability
print(f"MPS available: {torch.backends.mps.is_available()}")

# Use MPS device
device = torch.device("mps" if torch.backends.mps.is_available() else "cpu")
x = torch.tensor([1, 2, 3]).to(device)
\end{lstlisting}

\section{Getting Help}

If you encounter issues:

\begin{enumerate}
    \item Check the Troubleshooting Guide (separate document)
    \item Ask in the course forum/chat
    \item Come to office hours
    \item Use GitHub Codespaces as temporary backup (see backup guide)
\end{enumerate}

\section{Next Steps}

\begin{enumerate}
    \item Download Week 1 exercise notebooks from the course repository
    \item Place them in your \texttt{DeepLearning} folder
    \item Activate your virtual environment
    \item Start Jupyter Notebook
    \item Open \texttt{week1\_exercises\_starter.ipynb}
    \item You're ready to code!
\end{enumerate}

\begin{tipbox}
Create a habit: Always activate your virtual environment before working on course materials!
\end{tipbox}

\end{document}
