\documentclass[11pt,a4paper]{article}
\usepackage[margin=2.5cm]{geometry}
\usepackage{amsmath,amssymb}
\usepackage{enumitem}
\usepackage{hyperref}
\usepackage{listings}
\usepackage{xcolor}
\usepackage{tcolorbox}
\usepackage{fontawesome5}

% Code styling
\lstset{
    basicstyle=\ttfamily\small,
    keywordstyle=\color{blue},
    commentstyle=\color{gray},
    stringstyle=\color{red},
    showstringspaces=false,
    breaklines=true,
    frame=single,
    backgroundcolor=\color{gray!10}
}

% Custom boxes
\newtcolorbox{warningbox}{
    colback=red!5!white,
    colframe=red!75!black,
    title=\faExclamationTriangle\ Important
}

\newtcolorbox{infobox}{
    colback=blue!5!white,
    colframe=blue!75!black,
    title=\faInfoCircle\ Note
}

\newtcolorbox{tipbox}{
    colback=green!5!white,
    colframe=green!75!black,
    title=\faLightbulb\ Tip
}

\newtcolorbox{problembox}{
    colback=red!5!white,
    colframe=red!75!black,
    title=\faBug\ Problem
}

\newtcolorbox{solutionbox}{
    colback=green!5!white,
    colframe=green!75!black,
    title=\faCheck\ Solution
}

\title{Introduction to Deep Learning\\
Local Development Environment Setup Guide\\
\textbf{Windows}}
\author{MSc Artificial Intelligence}
\date{}

\begin{document}

\maketitle

\tableofcontents
\newpage

\section{Overview}

This guide will help you set up a local Python development environment on \textbf{Windows} for the Deep Learning course. You will:
\begin{itemize}
    \item Install Python 3.10 or later
    \item Create a virtual environment
    \item Install PyTorch and required packages
    \item Verify your installation with Jupyter Notebook
\end{itemize}

\begin{warningbox}
Please complete this setup \textbf{before} the first exercise session. If you encounter problems, consult the troubleshooting guide or use GitHub Codespaces as a temporary backup.
\end{warningbox}

\begin{infobox}
Estimated time: 30-45 minutes depending on your internet speed.
\end{infobox}

\section{Step 1: Check Existing Python Installation}

First, check if you already have Python 3.10 or later installed.

\subsection{Open Command Prompt}

Press \texttt{Win + R}, type \texttt{cmd}, and press Enter.

\subsection{Check Python Version}

In Command Prompt, run:
\begin{lstlisting}[language=bash]
python --version
\end{lstlisting}

\textbf{If you see \texttt{Python 3.10.x} or higher:}
\begin{itemize}
    \item Great! Skip to Step 3 (Setting Up Virtual Environment)
\end{itemize}

\textbf{If you see an error or older version:}
\begin{itemize}
    \item Continue to Step 2 to install Python
\end{itemize}

\section{Step 2: Install Python}

\subsection{Download Python}

\begin{enumerate}
    \item Go to \url{https://www.python.org/downloads/}
    \item Click on ``Download Python 3.11.x'' (or the latest 3.x version)
    \item The download should start automatically
    \item Save the installer file (e.g., \texttt{python-3.11.x-amd64.exe})
\end{enumerate}

\subsection{Run the Installer}

\begin{enumerate}
    \item Locate the downloaded installer and double-click to run it
    
    \item \textbf{CRITICAL STEP:} At the bottom of the installer window, check the box that says:
    \begin{center}
        \fbox{\texttt{Add Python 3.11 to PATH}}
    \end{center}
    
    \item Click ``Install Now''
    
    \item Wait for installation to complete (2-5 minutes)
    
    \item Click ``Close'' when finished
\end{enumerate}

\begin{warningbox}
\textbf{The ``Add Python to PATH'' checkbox is crucial!} Without this, Python commands won't work from the command line. If you forgot to check this box, uninstall Python and reinstall with this option checked.
\end{warningbox}

\subsection{Verify Installation}

\begin{enumerate}
    \item Open a \textbf{new} Command Prompt (close the old one if still open)
    \item Run:
    \begin{lstlisting}[language=bash]
python --version
    \end{lstlisting}
    
    \item You should see:
    \begin{lstlisting}
Python 3.11.x
    \end{lstlisting}
    
    \item Also verify pip is installed:
    \begin{lstlisting}[language=bash]
pip --version
    \end{lstlisting}
\end{enumerate}

\begin{tipbox}
If you still get ``'python' is not recognized'', see the Troubleshooting Guide section on PATH issues.
\end{tipbox}

\section{Step 3: Create Project Folder}

\subsection{Navigate to Documents}

In Command Prompt:
\begin{lstlisting}[language=bash]
cd C:\Users\%USERNAME%\Documents
\end{lstlisting}

\subsection{Create and Navigate to Project Folder}

\begin{lstlisting}[language=bash]
mkdir DeepLearning
cd DeepLearning
\end{lstlisting}

\begin{infobox}
You can create the folder anywhere, but avoid:
\begin{itemize}
    \item \texttt{C:\textbackslash Program Files} (permission issues)
    \item Folders with spaces or special characters in the name
    \item Very deep nested paths (Windows path length limits)
\end{itemize}
\end{infobox}

\section{Step 4: Create Virtual Environment}

A virtual environment keeps your project dependencies isolated.

\subsection{Create the Virtual Environment}

In your \texttt{DeepLearning} folder:
\begin{lstlisting}[language=bash]
python -m venv deep_learning_env
\end{lstlisting}

This creates a new folder called \texttt{deep\_learning\_env} containing:
\begin{itemize}
    \item Python interpreter
    \item pip package manager
    \item Space for installed packages
\end{itemize}

Wait 30-60 seconds for creation to complete.

\subsection{Activate the Virtual Environment}

\subsubsection{If using Command Prompt (cmd):}
\begin{lstlisting}[language=bash]
deep_learning_env\Scripts\activate.bat
\end{lstlisting}

\subsubsection{If using PowerShell:}
\begin{lstlisting}[language=bash]
deep_learning_env\Scripts\Activate.ps1
\end{lstlisting}

\begin{warningbox}
If PowerShell gives an error about execution policy, see the Troubleshooting Guide. Use Command Prompt (cmd) as an alternative.
\end{warningbox}

\subsection{Verify Activation}

After activation, you should see:
\begin{lstlisting}
(deep_learning_env) C:\Users\YourName\Documents\DeepLearning>
\end{lstlisting}

The \texttt{(deep\_learning\_env)} prefix indicates the virtual environment is active!

\begin{tipbox}
To deactivate later, simply type:
\begin{lstlisting}[language=bash]
deactivate
\end{lstlisting}
\end{tipbox}

\section{Step 5: Install Required Packages}

\begin{warningbox}
Make sure your virtual environment is activated! You should see \texttt{(deep\_learning\_env)} in your prompt.
\end{warningbox}

\subsection{Upgrade pip}

First, upgrade pip to the latest version:
\begin{lstlisting}[language=bash]
python -m pip install --upgrade pip
\end{lstlisting}

\subsection{Download requirements.txt}

\begin{enumerate}
    \item Download \texttt{requirements.txt} from the course repository
    \item Save it to your \texttt{DeepLearning} folder
    \item Verify it's there:
    \begin{lstlisting}[language=bash]
dir requirements.txt
    \end{lstlisting}
\end{enumerate}

\subsection{Install All Packages}

\begin{lstlisting}[language=bash]
pip install -r requirements.txt
\end{lstlisting}

\begin{infobox}
This will take 5-10 minutes as PyTorch is a large package (~700MB). You'll see progress bars for each package being downloaded and installed.
\end{infobox}

\subsection{Alternative: Manual Installation}

If you don't have \texttt{requirements.txt}, install packages individually:

\begin{lstlisting}[language=bash]
pip install torch torchvision torchaudio
pip install jupyter notebook
pip install matplotlib numpy pandas
pip install ipywidgets
\end{lstlisting}

\begin{tipbox}
If installation is very slow, see the Troubleshooting Guide for tips on using PyTorch's CDN.
\end{tipbox}

\section{Step 6: Verify Installation}

\subsection{Test Python and PyTorch}

\begin{enumerate}
    \item Start Python interpreter:
    \begin{lstlisting}[language=bash]
python
    \end{lstlisting}
    
    \item In the Python prompt (\texttt{>>>}), run:
    \begin{lstlisting}[language=Python]
import torch
print(f"PyTorch version: {torch.__version__}")
print(f"CUDA available: {torch.cuda.is_available()}")

# Create a test tensor
x = torch.tensor([1, 2, 3])
print(f"Test tensor: {x}")
    \end{lstlisting}
    
    \item You should see:
    \begin{lstlisting}
PyTorch version: 2.x.x
CUDA available: False
Test tensor: tensor([1, 2, 3])
    \end{lstlisting}
    
    \item Exit Python:
    \begin{lstlisting}[language=Python]
exit()
    \end{lstlisting}
\end{enumerate}

\begin{infobox}
\texttt{CUDA available: False} is normal if you don't have an NVIDIA GPU. PyTorch will use your CPU, which is fine for this course.
\end{infobox}

\subsection{Launch Jupyter Notebook}

\begin{enumerate}
    \item Make sure your virtual environment is still activated
    
    \item Run:
    \begin{lstlisting}[language=bash]
jupyter notebook
    \end{lstlisting}
    
    \item A browser window should open automatically showing the Jupyter interface
    
    \item You should see your \texttt{DeepLearning} folder contents
\end{enumerate}

\begin{tipbox}
If the browser doesn't open automatically:
\begin{itemize}
    \item Look for a URL in the terminal output
    \item It will look like: \texttt{http://localhost:8888/?token=...}
    \item Copy this URL and paste it into your browser
\end{itemize}
\end{tipbox}

\subsection{Create and Test a Notebook}

\begin{enumerate}
    \item In Jupyter, click ``New'' → ``Python 3 (ipykernel)''
    
    \item In the first cell, type:
    \begin{lstlisting}[language=Python]
import torch
import matplotlib.pyplot as plt
import numpy as np

print("Setup successful!")
print(f"PyTorch version: {torch.__version__}")

# Simple test
x = torch.randn(5)
print(f"Random tensor: {x}")
    \end{lstlisting}
    
    \item Press \texttt{Shift + Enter} to run the cell
    
    \item If you see ``Setup successful!'' and no errors, everything is working!
    
    \item Close the notebook (File → Close and Halt)
    
    \item Stop Jupyter by pressing \texttt{Ctrl + C} twice in the Command Prompt
\end{enumerate}

\section{Step 7: Daily Workflow}

Every time you want to work on the course:

\subsection{Starting Your Work Session}

\begin{enumerate}
    \item Open Command Prompt
    
    \item Navigate to your project folder:
    \begin{lstlisting}[language=bash]
cd C:\Users\%USERNAME%\Documents\DeepLearning
    \end{lstlisting}
    
    \item Activate virtual environment:
    \begin{lstlisting}[language=bash]
deep_learning_env\Scripts\activate
    \end{lstlisting}
    
    \item Start Jupyter:
    \begin{lstlisting}[language=bash]
jupyter notebook
    \end{lstlisting}
    
    \item Work in your notebooks
\end{enumerate}

\subsection{Ending Your Work Session}

\begin{enumerate}
    \item Save your notebooks (Ctrl + S)
    
    \item Close notebooks in Jupyter (File → Close and Halt)
    
    \item Stop Jupyter: \texttt{Ctrl + C} twice in Command Prompt
    
    \item Deactivate virtual environment:
    \begin{lstlisting}[language=bash]
deactivate
    \end{lstlisting}
    
    \item Close Command Prompt
\end{enumerate}

\section{Optional: IDE Setup}

While Jupyter Notebooks are great for exercises, you may want a full IDE for projects.

\subsection{VS Code (Recommended)}

\subsubsection{Installation}

\begin{enumerate}
    \item Download from \url{https://code.visualstudio.com/}
    \item Run the installer
    \item Check ``Add to PATH'' during installation
\end{enumerate}

\subsubsection{Setup for Python}

\begin{enumerate}
    \item Open VS Code
    
    \item Click Extensions icon (left sidebar) or press \texttt{Ctrl + Shift + X}
    
    \item Search for and install:
    \begin{itemize}
        \item ``Python'' by Microsoft
        \item ``Jupyter'' by Microsoft
    \end{itemize}
    
    \item Open your DeepLearning folder: File → Open Folder
    
    \item Select Python interpreter:
    \begin{itemize}
        \item Press \texttt{Ctrl + Shift + P}
        \item Type ``Python: Select Interpreter''
        \item Choose the one in \texttt{deep\_learning\_env\textbackslash Scripts\textbackslash python.exe}
    \end{itemize}
    
    \item You can now open and run .ipynb files directly in VS Code!
\end{enumerate}

\subsection{PyCharm Community (Alternative)}

\begin{enumerate}
    \item Download from \url{https://www.jetbrains.com/pycharm/download/#section=windows}
    \item Choose ``Community Edition'' (free)
    \item Run installer
    \item Open your DeepLearning folder
    \item Configure interpreter:
    \begin{itemize}
        \item File → Settings → Project → Python Interpreter
        \item Click gear icon → Add
        \item Choose ``Existing environment''
        \item Navigate to: \texttt{DeepLearning\textbackslash deep\_learning\_env\textbackslash Scripts\textbackslash python.exe}
    \end{itemize}
\end{enumerate}

\section{Optional: GPU Support}

\begin{infobox}
GPU support is \textbf{not required} for this course, but can speed up training significantly from Week 4 onwards. Skip this section if you don't have an NVIDIA GPU.
\end{infobox}

\subsection{Check if You Have NVIDIA GPU}

\subsubsection{Method 1: Device Manager}
\begin{enumerate}
    \item Press \texttt{Win + X} → Device Manager
    \item Expand ``Display adapters''
    \item Look for NVIDIA GeForce/RTX/Quadro
\end{enumerate}

\subsubsection{Method 2: Task Manager}
\begin{enumerate}
    \item Press \texttt{Ctrl + Shift + Esc}
    \item Click ``Performance'' tab
    \item Look for GPU in the left sidebar
\end{enumerate}

If you don't see NVIDIA GPU, you cannot use CUDA. Skip the rest of this section.

\subsection{Install NVIDIA Drivers}

\begin{enumerate}
    \item Go to \url{https://www.nvidia.com/download/index.aspx}
    \item Select your GPU model
    \item Download and install the latest driver
    \item Restart your computer
\end{enumerate}

\subsection{Install CUDA Toolkit}

\begin{enumerate}
    \item Go to \url{https://developer.nvidia.com/cuda-downloads}
    \item Select: Windows → x86\_64 → 11 → exe (local)
    \item Download CUDA Toolkit 11.8 or 12.1
    \item Run the installer (will take 10-15 minutes)
    \item Follow default options
\end{enumerate}

\subsection{Reinstall PyTorch with CUDA}

\begin{enumerate}
    \item Activate your virtual environment
    
    \item Uninstall current PyTorch:
    \begin{lstlisting}[language=bash]
pip uninstall torch torchvision torchaudio
    \end{lstlisting}
    
    \item Install PyTorch with CUDA 11.8:
    \begin{lstlisting}[language=bash]
pip install torch torchvision torchaudio --index-url https://download.pytorch.org/whl/cu118
    \end{lstlisting}
    
    \item Or for CUDA 12.1:
    \begin{lstlisting}[language=bash]
pip install torch torchvision torchaudio --index-url https://download.pytorch.org/whl/cu121
    \end{lstlisting}
\end{enumerate}

\subsection{Verify GPU Support}

\begin{lstlisting}[language=Python]
import torch

print(f"CUDA available: {torch.cuda.is_available()}")

if torch.cuda.is_available():
    print(f"GPU: {torch.cuda.get_device_name(0)}")
    print(f"CUDA version: {torch.version.cuda}")
    
    # Test GPU computation
    x = torch.randn(3, 3).cuda()
    print(f"Tensor on GPU: {x.device}")
\end{lstlisting}

Should print:
\begin{lstlisting}
CUDA available: True
GPU: NVIDIA GeForce RTX 3060
CUDA version: 11.8
Tensor on GPU: cuda:0
\end{lstlisting}

\begin{warningbox}
GPU setup can be tricky. If you have issues, stick with CPU for now. We can revisit GPU setup in Week 4 when it becomes more beneficial.
\end{warningbox}

\section{Troubleshooting Common Issues}

\subsection{Python Not Recognized}

\begin{problembox}
Error: \texttt{'python' is not recognized as an internal or external command}
\end{problembox}

\begin{solutionbox}
Python is not in your PATH:

\textbf{Option 1: Reinstall Python}
\begin{enumerate}
    \item Uninstall current Python
    \item Reinstall and check ``Add Python to PATH''
\end{enumerate}

\textbf{Option 2: Manually add to PATH}
\begin{enumerate}
    \item Search for ``Environment Variables'' in Start Menu
    \item Click ``Environment Variables''
    \item Under ``System variables'', find ``Path'', click ``Edit''
    \item Click ``New'' and add:
    \begin{itemize}
        \item \texttt{C:\textbackslash Users\textbackslash YourUsername\textbackslash AppData\textbackslash Local\textbackslash Programs\textbackslash Python\textbackslash Python311}
        \item \texttt{C:\textbackslash Users\textbackslash YourUsername\textbackslash AppData\textbackslash Local\textbackslash Programs\textbackslash Python\textbackslash Python311\textbackslash Scripts}
    \end{itemize}
    \item Click OK, close all terminals, open a new Command Prompt
\end{enumerate}
\end{solutionbox}

\subsection{Virtual Environment Won't Activate}

\begin{problembox}
After running activation command, no \texttt{(deep\_learning\_env)} prefix appears.
\end{problembox}

\begin{solutionbox}
\textbf{PowerShell Issue:}
\begin{enumerate}
    \item PowerShell may block scripts by default
    \item Run PowerShell as Administrator
    \item Execute:
    \begin{lstlisting}[language=bash]
Set-ExecutionPolicy -ExecutionPolicy RemoteSigned -Scope CurrentUser
    \end{lstlisting}
    \item Try activating again:
    \begin{lstlisting}[language=bash]
deep_learning_env\Scripts\Activate.ps1
    \end{lstlisting}
\end{enumerate}

\textbf{Alternative: Use Command Prompt}
\begin{lstlisting}[language=bash]
deep_learning_env\Scripts\activate.bat
\end{lstlisting}
\end{solutionbox}

\subsection{Permission Denied}

\begin{problembox}
Getting ``Access is denied'' or permission errors.
\end{problembox}

\begin{solutionbox}
\begin{enumerate}
    \item Don't create folder in \texttt{C:\textbackslash Program Files}
    \item Use your user directory: \texttt{C:\textbackslash Users\textbackslash YourUsername\textbackslash Documents}
    \item Run Command Prompt as regular user (not Administrator)
    \item Check antivirus isn't blocking Python
\end{enumerate}
\end{solutionbox}

\subsection{Package Installation Fails}

\begin{problembox}
\texttt{pip install} fails with timeout or connection errors.
\end{problembox}

\begin{solutionbox}
\textbf{Try PyTorch CDN (faster):}
\begin{lstlisting}[language=bash]
pip install torch torchvision torchaudio --index-url https://download.pytorch.org/whl/cpu
\end{lstlisting}

\textbf{Increase timeout:}
\begin{lstlisting}[language=bash]
pip install --default-timeout=1000 torch
\end{lstlisting}

\textbf{Check firewall/antivirus:}
\begin{itemize}
    \item Temporarily disable antivirus
    \item Add Python and pip to firewall exceptions
\end{itemize}
\end{solutionbox}

\subsection{Jupyter Notebook Won't Start}

\begin{problembox}
\texttt{jupyter notebook} fails or browser doesn't open.
\end{problembox}

\begin{solutionbox}
\begin{enumerate}
    \item Verify Jupyter is installed:
    \begin{lstlisting}[language=bash]
pip install jupyter notebook
    \end{lstlisting}
    
    \item Try different port:
    \begin{lstlisting}[language=bash]
jupyter notebook --port=8889
    \end{lstlisting}
    
    \item Check if port 8888 is in use:
    \begin{lstlisting}[language=bash]
netstat -ano | findstr :8888
    \end{lstlisting}
    
    \item Manually copy URL from terminal to browser
\end{enumerate}
\end{solutionbox}

\subsection{Module Not Found in Jupyter}

\begin{problembox}
\texttt{ModuleNotFoundError: No module named 'torch'} in Jupyter, even though you installed it.
\end{problembox}

\begin{solutionbox}
Jupyter is using wrong Python environment!

\begin{enumerate}
    \item Make sure virtual environment was activated BEFORE starting Jupyter
    
    \item Check Python path in notebook:
    \begin{lstlisting}[language=Python]
import sys
print(sys.executable)
    \end{lstlisting}
    Should point to \texttt{deep\_learning\_env}
    
    \item Register environment as kernel:
    \begin{lstlisting}[language=bash]
python -m ipykernel install --user --name=deep_learning_env
    \end{lstlisting}
    Then: Kernel → Change kernel → deep\_learning\_env
\end{enumerate}
\end{solutionbox}

\subsection{Long Path Errors}

\begin{problembox}
Errors about file paths being too long.
\end{problembox}

\begin{solutionbox}
\textbf{Enable long paths in Windows:}
\begin{enumerate}
    \item Open Registry Editor (Win+R, type \texttt{regedit})
    \item Navigate to:
    \begin{small}\texttt{HKEY\_LOCAL\_MACHINE\textbackslash SYSTEM\textbackslash CurrentControlSet\textbackslash Control\textbackslash FileSystem}\end{small}
    \item Set \texttt{LongPathsEnabled} to 1
    \item Restart computer
\end{enumerate}

\textbf{Or use shorter path:}
\begin{lstlisting}[language=bash]
cd C:\DL
\end{lstlisting}
\end{solutionbox}

\subsection{Antivirus Blocking Installation}

\begin{problembox}
Installation hangs or fails with cryptic errors.
\end{problembox}

\begin{solutionbox}
\begin{enumerate}
    \item Temporarily disable antivirus during installation
    \item Add Python directory to antivirus exceptions
    \item Add pip to exceptions
    \item Consider using Windows Defender instead of third-party antivirus
\end{enumerate}
\end{solutionbox}

\section{Getting Help}

If you've tried the troubleshooting steps and still have issues:

\begin{enumerate}
    \item Document your error:
    \begin{itemize}
        \item Copy full error message
        \item Note your Windows version
        \item Note Python version
        \item What you were trying to do
        \item What you've already tried
    \end{itemize}
    
    \item Get help:
    \begin{itemize}
        \item Post in course forum with documentation
        \item Email instructor with details
        \item Come to office hours
    \end{itemize}
    
    \item Temporary workaround:
    \begin{itemize}
        \item Use GitHub Codespaces (see separate guide)
        \item Continue with exercises while troubleshooting local setup
    \end{itemize}
\end{enumerate}

\section{Next Steps}

\begin{enumerate}
    \item Download Week 1 exercise notebooks from course repository
    \item Place them in your \texttt{DeepLearning} folder
    \item Activate virtual environment
    \item Start Jupyter Notebook
    \item Open \texttt{week1\_exercises\_starter.ipynb}
    \item You're ready to start coding!
\end{enumerate}

\begin{tipbox}
\textbf{Create a habit:} Always activate your virtual environment before working on course materials!
\end{tipbox}

\section{Quick Reference}

\subsection{Essential Commands}

\begin{lstlisting}[language=bash]
# Navigate to project
cd C:\Users\%USERNAME%\Documents\DeepLearning

# Activate virtual environment
deep_learning_env\Scripts\activate

# Start Jupyter
jupyter notebook

# Stop Jupyter
# Press Ctrl+C twice

# Deactivate virtual environment
deactivate
\end{lstlisting}

\subsection{Troubleshooting Quick Fixes}

\begin{itemize}
    \item \textbf{Python not recognized}: Reinstall with ``Add to PATH'' checked
    \item \textbf{Virtual env won't activate}: Use \texttt{activate.bat} instead of \texttt{Activate.ps1}
    \item \textbf{Module not found in Jupyter}: Make sure venv was activated before starting Jupyter
    \item \textbf{Slow installation}: Use PyTorch CDN (see Troubleshooting Guide)
\end{itemize}

\end{document}
