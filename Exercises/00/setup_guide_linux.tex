\documentclass[11pt,a4paper]{article}
\usepackage[margin=2.5cm]{geometry}
\usepackage{amsmath,amssymb}
\usepackage{enumitem}
\usepackage{hyperref}
\usepackage{listings}
\usepackage{xcolor}
\usepackage{tcolorbox}
\usepackage{fontawesome5}

% Code styling
\lstset{
    basicstyle=\ttfamily\small,
    keywordstyle=\color{blue},
    commentstyle=\color{gray},
    stringstyle=\color{red},
    showstringspaces=false,
    breaklines=true,
    frame=single,
    backgroundcolor=\color{gray!10}
}

% Custom boxes
\newtcolorbox{warningbox}{
    colback=red!5!white,
    colframe=red!75!black,
    title=\faExclamationTriangle\ Important
}

\newtcolorbox{infobox}{
    colback=blue!5!white,
    colframe=blue!75!black,
    title=\faInfoCircle\ Note
}

\newtcolorbox{tipbox}{
    colback=green!5!white,
    colframe=green!75!black,
    title=\faLightbulb\ Tip
}

\newtcolorbox{problembox}{
    colback=red!5!white,
    colframe=red!75!black,
    title=\faBug\ Problem
}

\newtcolorbox{solutionbox}{
    colback=green!5!white,
    colframe=green!75!black,
    title=\faCheck\ Solution
}

\title{Introduction to Deep Learning\\
Local Development Environment Setup Guide\\
\textbf{Linux}}
\author{MSc Artificial Intelligence}
\date{}

\begin{document}

\maketitle

\tableofcontents
\newpage

\section{Overview}

This guide will help you set up a local Python development environment on \textbf{Linux} for the Deep Learning course. You will:
\begin{itemize}
    \item Install Python 3.10 or later
    \item Create a virtual environment
    \item Install PyTorch and required packages
    \item Verify your installation with Jupyter Notebook
\end{itemize}

\begin{warningbox}
Please complete this setup \textbf{before} the first exercise session. If you encounter problems, consult the troubleshooting guide or use GitHub Codespaces as a temporary backup.
\end{warningbox}

\begin{infobox}
Estimated time: 20-30 minutes depending on your internet speed and distribution.\\
This guide covers Ubuntu/Debian. For other distributions (Fedora, Arch, etc.), adjust package manager commands accordingly.
\end{infobox}

\section{Step 1: Check Existing Python Installation}

Most Linux distributions come with Python pre-installed.

\subsection{Open Terminal}

Press \texttt{Ctrl + Alt + T} or search for ``Terminal'' in your applications.

\subsection{Check Python Version}

In terminal, run:
\begin{lstlisting}[language=bash]
python3 --version
\end{lstlisting}

\textbf{If you see \texttt{Python 3.10.x} or higher:}
\begin{itemize}
    \item Great! Skip to Step 3 (Setting Up Virtual Environment)
    \item But first, check if venv is installed (see Step 2.3)
\end{itemize}

\textbf{If you see an older version or error:}
\begin{itemize}
    \item Continue to Step 2 to install Python
\end{itemize}

\section{Step 2: Install Python}

\subsection{Ubuntu/Debian}

\subsubsection{Update Package List}

\begin{lstlisting}[language=bash]
sudo apt update
\end{lstlisting}

\subsubsection{Install Python 3.11}

\begin{lstlisting}[language=bash]
sudo apt install python3.11 python3.11-venv python3-pip
\end{lstlisting}

\begin{infobox}
If Python 3.11 is not available in your repositories, you may need to add the deadsnakes PPA:
\begin{lstlisting}[language=bash]
sudo add-apt-repository ppa:deadsnakes/ppa
sudo apt update
sudo apt install python3.11 python3.11-venv
\end{lstlisting}
\end{infobox}

\subsubsection{Verify Installation}

\begin{lstlisting}[language=bash]
python3.11 --version
\end{lstlisting}

\subsection{Fedora}

\begin{lstlisting}[language=bash]
sudo dnf install python3.11
\end{lstlisting}

\subsection{Arch Linux}

\begin{lstlisting}[language=bash]
sudo pacman -S python python-pip
\end{lstlisting}

\subsection{Check venv Module}

The \texttt{venv} module is required for creating virtual environments:

\begin{lstlisting}[language=bash]
python3 -m venv --help
\end{lstlisting}

\textbf{If you get an error:}
\begin{lstlisting}[language=bash]
# Ubuntu/Debian
sudo apt install python3.11-venv

# Fedora
sudo dnf install python3-virtualenv

# Arch
sudo pacman -S python-virtualenv
\end{lstlisting}

\section{Step 3: Create Project Folder}

\subsection{Navigate to Home Directory}

\begin{lstlisting}[language=bash]
cd ~
\end{lstlisting}

or

\begin{lstlisting}[language=bash]
cd ~/Documents
\end{lstlisting}

\subsection{Create and Navigate to Project Folder}

\begin{lstlisting}[language=bash]
mkdir DeepLearning
cd DeepLearning
\end{lstlisting}

\begin{tipbox}
You can verify your current location with:
\begin{lstlisting}[language=bash]
pwd
\end{lstlisting}
Should show: \texttt{/home/yourusername/DeepLearning}
\end{tipbox}

\section{Step 4: Create Virtual Environment}

A virtual environment keeps your project dependencies isolated.

\subsection{Create the Virtual Environment}

In your \texttt{DeepLearning} folder:
\begin{lstlisting}[language=bash]
python3 -m venv deep_learning_env
\end{lstlisting}

This creates a new folder called \texttt{deep\_learning\_env} containing:
\begin{itemize}
    \item Python interpreter
    \item pip package manager
    \item Space for installed packages
\end{itemize}

Wait 10-30 seconds for creation to complete.

\subsection{Activate the Virtual Environment}

\begin{lstlisting}[language=bash]
source deep_learning_env/bin/activate
\end{lstlisting}

\subsection{Verify Activation}

After activation, you should see:
\begin{lstlisting}
(deep_learning_env) username@hostname:~/DeepLearning$
\end{lstlisting}

The \texttt{(deep\_learning\_env)} prefix indicates the virtual environment is active!

\begin{tipbox}
To deactivate later, simply type:
\begin{lstlisting}[language=bash]
deactivate
\end{lstlisting}
\end{tipbox}

\begin{warningbox}
\textbf{Important:} You must use \texttt{source} to activate. Just running \texttt{deep\_learning\_env/bin/activate} without \texttt{source} will not work!
\end{warningbox}

\section{Step 5: Install Required Packages}

\begin{warningbox}
Make sure your virtual environment is activated! You should see \texttt{(deep\_learning\_env)} in your prompt.
\end{warningbox}

\subsection{Upgrade pip}

First, upgrade pip to the latest version:
\begin{lstlisting}[language=bash]
python3 -m pip install --upgrade pip
\end{lstlisting}

\subsection{Download requirements.txt}

\begin{enumerate}
    \item Download \texttt{requirements.txt} from the course repository
    \item Save it to your \texttt{DeepLearning} folder
    \item Verify it's there:
    \begin{lstlisting}[language=bash]
ls requirements.txt
    \end{lstlisting}
\end{enumerate}

\begin{tipbox}
You can download directly from terminal if you have the URL:
\begin{lstlisting}[language=bash]
wget https://url-to-requirements.txt
# or
curl -O https://url-to-requirements.txt
\end{lstlisting}
\end{tipbox}

\subsection{Install All Packages}

\begin{lstlisting}[language=bash]
pip install -r requirements.txt
\end{lstlisting}

\begin{infobox}
This will take 5-10 minutes as PyTorch is a large package (~700MB). You'll see progress bars for each package being downloaded and installed.
\end{infobox}

\subsection{Alternative: Manual Installation}

If you don't have \texttt{requirements.txt}, install packages individually:

\begin{lstlisting}[language=bash]
pip install torch torchvision torchaudio
pip install jupyter notebook
pip install matplotlib numpy pandas
pip install ipywidgets
\end{lstlisting}

\begin{warningbox}
\textbf{Never use sudo with pip!} This can break your system Python. Always use virtual environments.
\end{warningbox}

\section{Step 6: Verify Installation}

\subsection{Test Python and PyTorch}

\begin{enumerate}
    \item Start Python interpreter:
    \begin{lstlisting}[language=bash]
python3
    \end{lstlisting}
    
    \item In the Python prompt (\texttt{>>>}), run:
    \begin{lstlisting}[language=Python]
import torch
print(f"PyTorch version: {torch.__version__}")
print(f"CUDA available: {torch.cuda.is_available()}")

# Create a test tensor
x = torch.tensor([1, 2, 3])
print(f"Test tensor: {x}")
    \end{lstlisting}
    
    \item You should see:
    \begin{lstlisting}
PyTorch version: 2.x.x
CUDA available: False
Test tensor: tensor([1, 2, 3])
    \end{lstlisting}
    
    \item Exit Python:
    \begin{lstlisting}[language=Python]
exit()
    \end{lstlisting}
\end{enumerate}

\begin{infobox}
\texttt{CUDA available: False} is normal if you don't have an NVIDIA GPU. PyTorch will use your CPU, which is fine for this course.
\end{infobox}

\subsection{Launch Jupyter Notebook}

\begin{enumerate}
    \item Make sure your virtual environment is still activated
    
    \item Run:
    \begin{lstlisting}[language=bash]
jupyter notebook
    \end{lstlisting}
    
    \item A browser window should open automatically showing the Jupyter interface
    
    \item You should see your \texttt{DeepLearning} folder contents
\end{enumerate}

\begin{tipbox}
If the browser doesn't open automatically:
\begin{itemize}
    \item Look for a URL in the terminal output
    \item It will look like: \texttt{http://localhost:8888/?token=...}
    \item Copy this URL and paste it into your browser
\end{itemize}
\end{tipbox}

\subsection{Create and Test a Notebook}

\begin{enumerate}
    \item In Jupyter, click ``New'' → ``Python 3 (ipykernel)''
    
    \item In the first cell, type:
    \begin{lstlisting}[language=Python]
import torch
import matplotlib.pyplot as plt
import numpy as np

print("Setup successful!")
print(f"PyTorch version: {torch.__version__}")

# Simple test
x = torch.randn(5)
print(f"Random tensor: {x}")
    \end{lstlisting}
    
    \item Press \texttt{Shift + Enter} to run the cell
    
    \item If you see ``Setup successful!'' and no errors, everything is working!
    
    \item Close the notebook (File → Close and Halt)
    
    \item Stop Jupyter by pressing \texttt{Ctrl + C} twice in the terminal
\end{enumerate}

\section{Step 7: Daily Workflow}

Every time you want to work on the course:

\subsection{Starting Your Work Session}

\begin{enumerate}
    \item Open Terminal (\texttt{Ctrl + Alt + T})
    
    \item Navigate to your project folder:
    \begin{lstlisting}[language=bash]
cd ~/DeepLearning
    \end{lstlisting}
    
    \item Activate virtual environment:
    \begin{lstlisting}[language=bash]
source deep_learning_env/bin/activate
    \end{lstlisting}
    
    \item Start Jupyter:
    \begin{lstlisting}[language=bash]
jupyter notebook
    \end{lstlisting}
    
    \item Work in your notebooks
\end{enumerate}

\subsection{Ending Your Work Session}

\begin{enumerate}
    \item Save your notebooks (\texttt{Ctrl + S})
    
    \item Close notebooks in Jupyter (File → Close and Halt)
    
    \item Stop Jupyter: \texttt{Ctrl + C} twice in terminal
    
    \item Deactivate virtual environment:
    \begin{lstlisting}[language=bash]
deactivate
    \end{lstlisting}
    
    \item Close terminal
\end{enumerate}

\begin{tipbox}
Create a shell alias to make activation easier:
\begin{lstlisting}[language=bash]
# Add to ~/.bashrc or ~/.zshrc
alias dlenv='cd ~/DeepLearning && source deep_learning_env/bin/activate'
\end{lstlisting}
Then you can just type \texttt{dlenv} to navigate and activate!
\end{tipbox}

\section{Optional: IDE Setup}

While Jupyter Notebooks are great for exercises, you may want a full IDE for projects.

\subsection{VS Code (Recommended)}

\subsubsection{Installation}

\textbf{Ubuntu/Debian:}
\begin{lstlisting}[language=bash]
# Download .deb package
wget -O vscode.deb 'https://code.visualstudio.com/sha/download?build=stable&os=linux-deb-x64'

# Install
sudo apt install ./vscode.deb
\end{lstlisting}

\textbf{Fedora:}
\begin{lstlisting}[language=bash]
sudo rpm --import https://packages.microsoft.com/keys/microsoft.asc
sudo dnf install code
\end{lstlisting}

\textbf{Arch:}
\begin{lstlisting}[language=bash]
yay -S visual-studio-code-bin
\end{lstlisting}

\subsubsection{Setup for Python}

\begin{enumerate}
    \item Launch VS Code: \texttt{code}
    
    \item Click Extensions icon (left sidebar) or press \texttt{Ctrl + Shift + X}
    
    \item Search for and install:
    \begin{itemize}
        \item ``Python'' by Microsoft
        \item ``Jupyter'' by Microsoft
    \end{itemize}
    
    \item Open your DeepLearning folder: File → Open Folder
    
    \item Select Python interpreter:
    \begin{itemize}
        \item Press \texttt{Ctrl + Shift + P}
        \item Type ``Python: Select Interpreter''
        \item Choose the one in \texttt{deep\_learning\_env/bin/python}
    \end{itemize}
    
    \item You can now open and run .ipynb files directly in VS Code!
\end{enumerate}

\subsection{PyCharm Community (Alternative)}

\subsubsection{Installation}

\textbf{Ubuntu (Snap):}
\begin{lstlisting}[language=bash]
sudo snap install pycharm-community --classic
\end{lstlisting}

\textbf{Manual Download:}
\begin{enumerate}
    \item Go to \url{https://www.jetbrains.com/pycharm/download/#section=linux}
    \item Download Community Edition
    \item Extract and run
\end{enumerate}

\subsubsection{Configuration}

\begin{enumerate}
    \item Open PyCharm
    \item Open your DeepLearning folder
    \item Configure interpreter:
    \begin{itemize}
        \item File → Settings → Project → Python Interpreter
        \item Click gear icon → Add
        \item Choose ``Existing environment''
        \item Navigate to: \texttt{~/DeepLearning/deep\_learning\_env/bin/python}
    \end{itemize}
\end{enumerate}

\section{Optional: GPU Support}

\begin{infobox}
GPU support is \textbf{not required} for this course, but can speed up training significantly from Week 4 onwards. Skip this section if you don't have an NVIDIA GPU.
\end{infobox}

\subsection{Check if You Have NVIDIA GPU}

\begin{lstlisting}[language=bash]
lspci | grep -i nvidia
\end{lstlisting}

If you see output listing an NVIDIA GPU, you can set up CUDA. Otherwise, skip this section.

\subsection{Install NVIDIA Drivers}

\textbf{Ubuntu:}
\begin{lstlisting}[language=bash]
# Check recommended driver
ubuntu-drivers devices

# Install recommended driver
sudo ubuntu-drivers autoinstall

# Or install specific version
sudo apt install nvidia-driver-535

# Reboot
sudo reboot
\end{lstlisting}

\textbf{Fedora:}
\begin{lstlisting}[language=bash]
sudo dnf install akmod-nvidia
sudo reboot
\end{lstlisting}

\textbf{Arch:}
\begin{lstlisting}[language=bash]
sudo pacman -S nvidia nvidia-utils
sudo reboot
\end{lstlisting}

\subsection{Verify Driver Installation}

After reboot:
\begin{lstlisting}[language=bash]
nvidia-smi
\end{lstlisting}

Should show your GPU information.

\subsection{Install CUDA Toolkit}

\textbf{Ubuntu:}
\begin{lstlisting}[language=bash]
# Add CUDA repository
wget https://developer.download.nvidia.com/compute/cuda/repos/ubuntu2204/x86_64/cuda-keyring_1.1-1_all.deb
sudo dpkg -i cuda-keyring_1.1-1_all.deb
sudo apt update

# Install CUDA
sudo apt install cuda-toolkit-11-8
\end{lstlisting}

\textbf{Other distributions:}
See \url{https://developer.nvidia.com/cuda-downloads}

\subsection{Reinstall PyTorch with CUDA}

\begin{enumerate}
    \item Activate your virtual environment
    
    \item Uninstall current PyTorch:
    \begin{lstlisting}[language=bash]
pip uninstall torch torchvision torchaudio
    \end{lstlisting}
    
    \item Install PyTorch with CUDA 11.8:
    \begin{lstlisting}[language=bash]
pip install torch torchvision torchaudio --index-url https://download.pytorch.org/whl/cu118
    \end{lstlisting}
    
    \item Or for CUDA 12.1:
    \begin{lstlisting}[language=bash]
pip install torch torchvision torchaudio --index-url https://download.pytorch.org/whl/cu121
    \end{lstlisting}
\end{enumerate}

\subsection{Verify GPU Support}

\begin{lstlisting}[language=Python]
import torch

print(f"CUDA available: {torch.cuda.is_available()}")

if torch.cuda.is_available():
    print(f"GPU: {torch.cuda.get_device_name(0)}")
    print(f"CUDA version: {torch.version.cuda}")
    
    # Test GPU computation
    x = torch.randn(3, 3).cuda()
    print(f"Tensor on GPU: {x.device}")
\end{lstlisting}

Should print:
\begin{lstlisting}
CUDA available: True
GPU: NVIDIA GeForce RTX 3060
CUDA version: 11.8
Tensor on GPU: cuda:0
\end{lstlisting}

\begin{warningbox}
GPU setup can be tricky. If you have issues, stick with CPU for now. We can revisit GPU setup in Week 4 when it becomes more beneficial.
\end{warningbox}

\section{Troubleshooting Common Issues}

\subsection{Python Not Found}

\begin{problembox}
Error: \texttt{python3: command not found}
\end{problembox}

\begin{solutionbox}
\begin{enumerate}
    \item Install Python (see Step 2)
    \item Check if it's installed but not in PATH:
    \begin{lstlisting}[language=bash]
which python3
whereis python3
    \end{lstlisting}
    \item Try with version number:
    \begin{lstlisting}[language=bash]
python3.11 --version
    \end{lstlisting}
\end{enumerate}
\end{solutionbox}

\subsection{venv Module Not Found}

\begin{problembox}
Error: \texttt{The virtual environment was not created successfully because ensurepip is not available}
\end{problembox}

\begin{solutionbox}
Install venv module:
\begin{lstlisting}[language=bash]
# Ubuntu/Debian
sudo apt install python3.11-venv

# Fedora
sudo dnf install python3-virtualenv

# Arch
sudo pacman -S python-virtualenv
\end{lstlisting}
\end{solutionbox}

\subsection{Virtual Environment Won't Activate}

\begin{problembox}
After running activation, no \texttt{(deep\_learning\_env)} prefix appears.
\end{problembox}

\begin{solutionbox}
\begin{enumerate}
    \item Make sure you use \texttt{source}:
    \begin{lstlisting}[language=bash]
source deep_learning_env/bin/activate
    \end{lstlisting}
    
    \item Check if venv was created successfully:
    \begin{lstlisting}[language=bash]
ls deep_learning_env/bin/
    \end{lstlisting}
    Should see \texttt{activate} file
    
    \item Try recreating:
    \begin{lstlisting}[language=bash]
rm -rf deep_learning_env
python3 -m venv deep_learning_env
source deep_learning_env/bin/activate
    \end{lstlisting}
\end{enumerate}
\end{solutionbox}

\subsection{Permission Denied}

\begin{problembox}
Getting ``Permission denied'' errors when creating venv or installing packages.
\end{problembox}

\begin{solutionbox}
\textbf{NEVER use sudo with pip or venv!}

\begin{enumerate}
    \item Make sure you own the directory:
    \begin{lstlisting}[language=bash]
cd ~/Documents
mkdir DeepLearning
cd DeepLearning
    \end{lstlisting}
    
    \item If you accidentally used sudo, fix permissions:
    \begin{lstlisting}[language=bash]
sudo chown -R $USER:$USER ~/Documents/DeepLearning
    \end{lstlisting}
    
    \item Delete venv and recreate without sudo:
    \begin{lstlisting}[language=bash]
rm -rf deep_learning_env
python3 -m venv deep_learning_env
    \end{lstlisting}
\end{enumerate}
\end{solutionbox}

\subsection{Externally-Managed-Environment Error}

\begin{problembox}
Error: \texttt{externally-managed-environment} when installing packages (Ubuntu 23.04+).
\end{problembox}

\begin{solutionbox}
This is exactly why we use virtual environments!

\begin{enumerate}
    \item Make sure you created a virtual environment
    \item Make sure it's activated (see \texttt{(deep\_learning\_env)} prefix)
    \item If activated and still getting error, recreate venv:
    \begin{lstlisting}[language=bash]
deactivate
rm -rf deep_learning_env
python3 -m venv deep_learning_env
source deep_learning_env/bin/activate
pip install -r requirements.txt
    \end{lstlisting}
\end{enumerate}

\textbf{Do NOT use} \texttt{--break-system-packages}! Use virtual environments instead.
\end{solutionbox}

\subsection{Package Installation Fails}

\begin{problembox}
\texttt{pip install} fails with timeout or connection errors.
\end{problembox}

\begin{solutionbox}
\textbf{Try PyTorch CDN (faster):}
\begin{lstlisting}[language=bash]
pip install torch torchvision torchaudio --index-url https://download.pytorch.org/whl/cpu
\end{lstlisting}

\textbf{Increase timeout:}
\begin{lstlisting}[language=bash]
pip install --default-timeout=1000 torch
\end{lstlisting}

\textbf{Check internet connection:}
\begin{lstlisting}[language=bash]
ping pypi.org
\end{lstlisting}
\end{solutionbox}

\subsection{Jupyter Notebook Won't Start}

\begin{problembox}
\texttt{jupyter notebook} fails or browser doesn't open.
\end{problembox}

\begin{solutionbox}
\begin{enumerate}
    \item Verify Jupyter is installed:
    \begin{lstlisting}[language=bash]
pip install jupyter notebook
    \end{lstlisting}
    
    \item Try different port:
    \begin{lstlisting}[language=bash]
jupyter notebook --port=8889
    \end{lstlisting}
    
    \item Check if port 8888 is in use:
    \begin{lstlisting}[language=bash]
lsof -i :8888
    \end{lstlisting}
    
    \item Manually copy URL from terminal to browser
\end{enumerate}
\end{solutionbox}

\subsection{Module Not Found in Jupyter}

\begin{problembox}
\texttt{ModuleNotFoundError: No module named 'torch'} in Jupyter.
\end{problembox}

\begin{solutionbox}
Jupyter is using wrong Python environment!

\begin{enumerate}
    \item Activate virtual environment BEFORE starting Jupyter
    
    \item Check Python path in notebook:
    \begin{lstlisting}[language=Python]
import sys
print(sys.executable)
    \end{lstlisting}
    Should point to \texttt{deep\_learning\_env}
    
    \item Register environment as kernel:
    \begin{lstlisting}[language=bash]
python -m ipykernel install --user --name=deep_learning_env
    \end{lstlisting}
    Then: Kernel → Change kernel → deep\_learning\_env
\end{enumerate}
\end{solutionbox}

\subsection{SSL Certificate Errors}

\begin{problembox}
\texttt{SSL: CERTIFICATE\_VERIFY\_FAILED} when installing packages.
\end{problembox}

\begin{solutionbox}
\textbf{Update certificates:}
\begin{lstlisting}[language=bash]
# Ubuntu/Debian
sudo apt install ca-certificates
sudo update-ca-certificates

# Fedora
sudo dnf install ca-certificates
\end{lstlisting}

\textbf{Temporary workaround (not recommended):}
\begin{lstlisting}[language=bash]
pip install --trusted-host pypi.org --trusted-host files.pythonhosted.org torch
\end{lstlisting}
\end{solutionbox}

\section{Getting Help}

If you've tried the troubleshooting steps and still have issues:

\begin{enumerate}
    \item Document your error:
    \begin{itemize}
        \item Copy full error message
        \item Note your Linux distribution and version
        \item Note Python version (\texttt{python3 --version})
        \item What you were trying to do
        \item What you've already tried
    \end{itemize}
    
    \item Get help:
    \begin{itemize}
        \item Post in course forum with documentation
        \item Email instructor with details
        \item Come to office hours
    \end{itemize}
    
    \item Temporary workaround:
    \begin{itemize}
        \item Use GitHub Codespaces (see separate guide)
        \item Continue with exercises while troubleshooting local setup
    \end{itemize}
\end{enumerate}

\section{Next Steps}

\begin{enumerate}
    \item Download Week 1 exercise notebooks from course repository
    \item Place them in your \texttt{DeepLearning} folder
    \item Activate virtual environment
    \item Start Jupyter Notebook
    \item Open \texttt{week1\_exercises\_starter.ipynb}
    \item You're ready to start coding!
\end{enumerate}

\begin{tipbox}
\textbf{Create a habit:} Always activate your virtual environment before working on course materials!
\end{tipbox}

\section{Quick Reference}

\subsection{Essential Commands}

\begin{lstlisting}[language=bash]
# Navigate to project
cd ~/DeepLearning

# Activate virtual environment
source deep_learning_env/bin/activate

# Start Jupyter
jupyter notebook

# Stop Jupyter
# Press Ctrl+C twice

# Deactivate virtual environment
deactivate
\end{lstlisting}

\subsection{Troubleshooting Quick Fixes}

\begin{itemize}
    \item \textbf{python3: command not found}: Install Python (Step 2)
    \item \textbf{venv module not found}: Install \texttt{python3-venv} package
    \item \textbf{Virtual env won't activate}: Use \texttt{source} command
    \item \textbf{Module not found in Jupyter}: Activate venv before starting Jupyter
    \item \textbf{Permission denied}: Don't use \texttt{sudo}, check folder ownership
    \item \textbf{externally-managed-environment error}: Use virtual environment!
\end{itemize}

\end{document}
